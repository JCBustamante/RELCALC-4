\chapter{Séries de Fourier et équations différentielles partielles} \label{FS:chapter}

%%%%%%%%%%%%%%%%%%%%%%%%%%%%%%%%%%%%%%%%%%%%%%%%%%%%%%%%%%%%%%%%%%%%%%%%%%%%%%

\section{Équations avec conditions au bord} \label{bvp:section}

%\sectionnotes{2 lectures\EPref{, similar to \S3.8 in \cite{EP}}\BDref{,
%\S10.1 and \S11.1 in \cite{BD}}}

\subsection{Valeurs au bord}

Avant de s'attaquer aux séries de Fourier, on va étudier les problèmes \emph{à valeurs au bord\index{boundary value problem}}
 (ou \emph{à valeurs limites\index{endpoint problem}}).  On considère
\begin{equation*}
x'' + \lambda x = 0, \quad x(a) = 0, \quad x(b) = 0
\end{equation*}
pour une certaine constante $\lambda$, où $x(t)$ prend ses valeurs  $t$ dans l'intervalle 
$[a,b]$.
Précédemment, on a spécifié la valeur de la solution et de sa dérivée en un point. Maintenant, on veut spécifier la valeur de la solution à deux points différents. Lorsque $x=0$ est une solution, l'existence de la solution n'est pas problématique. Par contre, l'unicité de la solution est une autre question. La solution générale de  $x'' + \lambda x = 0$ a deux constantes arbitraires\footnote{%
On regarde la \subsectionref{subsection:fourfundamental}, ou l'\exampleref {example:expsecondorder} et l'\exampleref{example:sincossecondorder}.}.
C'est naturel (mais faux) de croire qu'exiger deux solutions garantit l'unicité. 

\begin{example}
On prend $\lambda = 1$,
$a=0$, $b=\pi$.  Ceci donne: 
\begin{equation*}
x'' + x = 0, \quad x(0) = 0, \quad x(\pi) = 0.
\end{equation*}
Alors, $x = \sin t$ est une autre solution  (qui s'ajoute à $x=0$) satisfaisant aux conditions limites. 
Il y a plus. Écrivons la solution générale de l'équation différentielle, qui est  $x= A \cos t + B \sin t$.
La condition $x(0) = 0$ implique que $A=0$.  La contrainte $x(\pi) = 0$ ne donne pas plus d'information lorsque $x = B \sin t$ satisfait déjà aux conditions limites. 
Par conséquent, il y a une infinité de solutions de la forme  $x = B \sin t$,
où $B$ est une constante arbitraire.  
\end{example}

\begin{example}
Posons maintenant $\lambda = 2$: 
\begin{equation*}
x'' + 2 x = 0, \quad x(0) = 0, \quad x(\pi) = 0.
\end{equation*}
Alors, la solution générale est
$x= A \cos ( \sqrt{2}\,t) + B \sin ( \sqrt{2}\,t)$.  La contrainte $x(0) = 0$ implique encore que $A = 0$.  On applique la seconde condition pour trouver 
$0=x(\pi) = B \sin ( \sqrt{2}\,\pi)$.
Lorsque $\sin ( \sqrt{2}\,\pi) \not= 0$, on obtient
$B = 0$.  
Donc, $x=0$ est l'unique solution à ce problème.
\end{example}

Que se passe-t-il ?  On aimerait savoir quelles constantes  $\lambda$ permettent une solution non nulle et l'on s'intéressera à trouver ces solutions. Ce problème est analogue  à trouver les valeurs propres et les vecteurs propres des matrices.

\subsection{Problèmes de valeurs propres}

Pour la théorie de base des séries de Fourier, on a besoin
des trois problèmes de valeurs propres suivants : 

\begin{equation} \label{bv:eq1}
x'' + \lambda x = 0, \quad x(a) = 0, \quad x(b) = 0 ,
\end{equation}
\begin{equation} \label{bv:eq2}
x'' + \lambda x = 0, \quad x'(a) = 0, \quad x'(b) = 0
\end{equation}
et
\begin{equation} \label{bv:eq3}
x'' + \lambda x = 0, \quad x(a) = x(b), \quad x'(a) = x'(b) .
\end{equation}
Un nombre $\lambda$ est appelé une
\emph{valeur propre \index{eigenvalue of a boundary value problem}}
de \eqref{bv:eq1}
(resp.\ \eqref{bv:eq2} ou \eqref{bv:eq3}) si et seulement si il existe une solution non nulle (non identiquement nulle) à \eqref{bv:eq1}
(resp.\ \eqref{bv:eq2} ou \eqref{bv:eq3})
donnée par un $\lambda$ spécifique. Une solution non nulle est appelée 
\emph{\myindex{fonction propre correspondante}}\index{corresponding eigenfunction}.

On note la similarité avec les valeurs propres et les vecteurs propres de matrices. La similarité n'est pas une simple coïncidence.  Par exemple, soit $L = -\frac{d^2}{{dt}^2}$ un opérateur linéaire.  Une fonction propre pour $L$ sera une fonction $x(t)$ non nulle telle que $Lx = \lambda x$, pour un certain $\lambda$, qui sera alors une valeur propre de l'opérateur $L$.

En d'autres mots, on pense à l'équation différentielle comme à un opérateur différentiel, et à une fonction $x(t)$
comme à un vecteur avec une infinité de composantes (une pour chaque $t$).
Une fonction propre sera une fonction non nulle $x$ qui satisfait à 
$(L- \lambda)x = 0$. Il y a beaucoup de formalisme provenant de l'algèbre linéaire qui s'applique ici, mais on ne poursuivra pas cette réflexion trop loin.  

\begin{example} \label{bvp:eig1ex}
Trouvons les valeurs propres et les fonctions propres de 
\begin{equation*}
x'' + \lambda x = 0, \quad x(0) = 0, \quad x(\pi) = 0 .
\end{equation*}

%For reasons that will be clear from the computations,
On doit considérer chacun des cas suivants séparément, $\lambda > 0$, $\lambda = 0$, $\lambda < 0$, puisque la solution générale est différente dans les trois cas.
D'abord, on suppose que $\lambda > 0$.  Alors, la solution générale à $x''+\lambda x = 0$ est
\begin{equation*}
x = A \cos ( \sqrt{\lambda}\, t) + B \sin ( \sqrt{\lambda}\, t).
\end{equation*}
La condition $x(0) = 0$ implique immédiatement que $A = 0$.
Ensuite,
\begin{equation*}
0 = x(\pi) = B \sin ( \sqrt{\lambda}\, \pi ) .
\end{equation*}
Si $B$ est nul, alors $x$ est une solution nulle. Alors, pour obtenir une solution non nulle, on doit avoir que $\sin ( \sqrt{\lambda}\, \pi) = 0$.  Donc,
$\sqrt{\lambda}\, \pi$ doit être un entier multiple de $\pi$.  En d'autres mots,
 $\sqrt{\lambda} = k$ pour un entier positif $k$.
Ainsi, les valeurs propres positives sont 
$k^2$ pour tous les entiers $k \geq 1$.  Les fonctions propres correspondantes peuvent être vues comme $x=\sin (k t)$.  Tout comme les vecteurs propres, les multiples d'une fonction propre sont des fonctions propres. 

On suppose que $\lambda = 0$.  Dans ce cas, l'équation est $x'' = 0$,
et sa solution générale est $x = At + B$.  La condition  $x(0) = 0$ implique que 
 $B=0$, et $x(\pi) = 0$ implique que $A = 0$, ce qui signifie que $\lambda
= 0$ n'est \emph{pas} une valeur propre.

Finalement, on suppose que $\lambda < 0$. Dans ce cas, on a la solution générale\footnote{%
Rappelons que 
$\cosh s = \frac{1}{2}(e^s+e^{-s})$
et
$\sinh s = \frac{1}{2}(e^s-e^{-s})$.  En exercice, faites à nouveau les calculs avec la solution générale écrite comme 
$x = A e^{\sqrt{-\lambda}\, t} + B e^{-\sqrt{-\lambda}\, t}$ (pour différents  $A$ et $B$ évidemment).}
\begin{equation*}
x = A \cosh ( \sqrt{-\lambda}\, t) + B \sinh ( \sqrt{-\lambda}\, t ) .
\end{equation*}
Si $x(0) = 0$, ça implique que $A = 0$ (on se rappelle que $\cosh 0 = 1$ et $\sinh 0 =
0$).  Alors, la solution devrait être $x = B \sinh ( \sqrt{-\lambda}\, t )$ et devrait satisfaire à 
$x(\pi) = 0$, ce qui est possible uniquement si $B$ est nul. Pourquoi? Parce que 
$\sinh \xi$ est seulement zéro lorsque $\xi=0$.  On devrait calculer sinh pour voir ce fait. On peut aussi le voir de la définition de sinh. On obtient $0 = \sinh \xi = \frac{e^\xi -
e^{-\xi}}{2}$. Donc, $e^\xi = e^{-\xi}$, ce qui implique que $\xi = -\xi$, ce qui est seulement vrai si $\xi=0$. Par conséquent, il n'y a pas de valeurs propres négatives.  

En somme, les valeurs propres et les fonctions propres correspondantes sont 
\begin{equation*}
\lambda_k = k^2 \quad \text{avec fonction propre} \quad x_k = \sin (k t)
\quad \text{pour tous les entiers } k \geq 1 .
\end{equation*}
\end{example}

\begin{example}
Calculons les valeurs propres et les fonctions propres de 
\begin{equation*}
x'' + \lambda x = 0, \quad x'(0) = 0, \quad x'(\pi) = 0 .
\end{equation*}

Encore une fois, on doit considérer séparément  les trois cas suivants: $\lambda > 0$, $\lambda = 0$, $\lambda
< 0$.
D'abord, on suppose que $\lambda > 0$.
La solution générale est alors 
$x = A \cos ( \sqrt{\lambda}\, t) + B \sin ( \sqrt{\lambda}\, t)$.  Sa dérivée est: 
\begin{equation*}
x' = -A\sqrt{\lambda}\, \sin ( \sqrt{\lambda}\, t) + B\sqrt{\lambda}\,
\cos (\sqrt{\lambda}\, t) .
\end{equation*}
La condition $x'(0) = 0$ implique immédiatement que  $B = 0$.
Ensuite,
\begin{equation*}
0 = x'(\pi) = -A\sqrt{\lambda}\, \sin ( \sqrt{\lambda}\, \pi) .
\end{equation*}
Encore une fois,  $A$ ne peut pas être nul si  $\lambda$ est une valeur propre, et $\sin ( \sqrt{\lambda}\, \pi)$ est identiquement nul lorsque
$\sqrt{\lambda} = k$,  où $k$ est un nombre entier positif.
Donc, les valeurs propres positives sont encore 
$k^2$ pour tous les entiers $k \geq 1$, et les fonctions propres correspondantes peuvent être prises comme  $x=\cos (k t)$.

Maintenant, on suppose que  $\lambda = 0$.  Dans ce cas, l'équation est $x'' = 0$,
et la solution générale est $x = At + B$ so $x' = A$.  La condition 
$x'(0) = 0$ implique que 
$A=0$.  La condition $x'(\pi) = 0$ implique aussi que  $A=0$.
Ainsi, $B$ pourrait être n'importe quoi (choisissons 1). Alors, $\lambda = 0$
est une valeur propre, et $x=1$ est la fonction propre correspondante. 

Finalement, soit $\lambda < 0$.  Dans ce cas, la solution générale est 
$x = A \cosh ( \sqrt{-\lambda}\, t) + B \sinh ( \sqrt{-\lambda}\, t)$
et
\begin{equation*}
x' = A\sqrt{-\lambda}\, \sinh ( \sqrt{-\lambda}\, t)
+ B\sqrt{-\lambda}\, \cosh ( \sqrt{-\lambda}\, t ) .
\end{equation*}
On a déjà vu (avec les rôles de  $A$ et de $B$ inversés) que, pour que cette expression soit  nulle à $t=0$ et $t=\pi$, on doit avoir $A=B=0$. Ainsi, il n'y a pas de valeur propre négative. 

En somme, les valeurs propres positives et leurs fonctions propres correspondantes sont
\begin{equation*}
\lambda_k = k^2 \quad \text{avec fonction propre} \quad x_k = \cos (k t)
\quad \text{pour tous les entiers } k \geq 1 ,
\end{equation*}
et, de plus, il y a une autre valeur propre: 
\begin{equation*}
\lambda_0 = 0 \qquad \text{avec une fonction propre} \qquad x_0 = 1.
\end{equation*}
\end{example}

Le problème suivant est celui qui a mené à la série de Fourier générale.

\begin{example} \label{bvp-periodic:example}
Calculons les valeurs propres et les fonctions propres de 
\begin{equation*}
x'' + \lambda x = 0, \quad x(-\pi) = x(\pi), \quad x'(-\pi) = x'(\pi) .
\end{equation*}
On n'a pas de valeur spécifique de fonction ou de dérivée aux extrémités, mais elles sont les mêmes au début et à la fin de l'intervalle. 

On va passer le cas $\lambda < 0$.  Les calculs sont les mêmes que précédemment, et l'on trouve encore qu'il n'y a pas de valeur propre négative. 

Pour $\lambda = 0$, la solution générale est $x = At + B$.  La condition
$x(-\pi) = x(\pi)$ implique que $A=0$ ($A\pi + B = -A\pi +B$ implique que $A=0$).
La seconde condition $x'(-\pi) = x'(\pi)$ ne dit rien à propos de $B$, et donc
$\lambda=0$ est une valeur propre avec les fonctions propres correspondantes $x=1$.

Pour $\lambda > 0$, on a que 
$x = A \cos ( \sqrt{\lambda}\, t ) + B \sin ( \sqrt{\lambda}\, t)$.
Maintenant,
\begin{equation*}
\underbrace{A \cos (-\sqrt{\lambda}\, \pi) + B \sin (-\sqrt{\lambda}\,
\pi)}_{x(-\pi)}
=
\underbrace{A \cos (  \sqrt{\lambda}\, \pi ) + B \sin ( \sqrt{\lambda}\,
\pi)}_{x(\pi)} .
\end{equation*}
On se rappelle que $\cos (- \theta) = \cos (\theta)$ et
$\sin (-\theta) = - \sin (\theta)$. Donc,
\begin{equation*}
A \cos (\sqrt{\lambda}\, \pi) - B \sin ( \sqrt{\lambda}\, \pi)
=
A \cos (\sqrt{\lambda}\, \pi) + B \sin ( \sqrt{\lambda}\, \pi).
\end{equation*}
Donc, ou bien $B=0$, ou bien $\sin ( \sqrt{\lambda}\, \pi) = 0$.
De manière semblable à l'exemple précédent (exercice), si l'on dérive $x$ et qu'on le remplace dans la seconde condition, on trouve que $A=0$ ou que $\sin ( \sqrt{\lambda}\, \pi) = 0$.
Ainsi, à moins que $A$ et $B$ soient tous deux nuls (ce qu'on ne veut pas), on doit avoir $\sin ( \sqrt{\lambda}\, \pi ) = 0$.  Donc, $\sqrt{\lambda}$
est un entier, et les valeurs propres sont encore $\lambda = k^2$ pour un entier
 $k \geq 1$. Dans ce cas, toutefois,  
$x = A \cos (k t) + B \sin (k t)$ est une fonction propre pour tout $A$ et pour tout $B$.
Par conséquent, on a deux fonctions propres linéairement indépendantes: $\sin (kt)$ et $\cos (kt)$.

On se rappelle que, pour une matrice, on peut aussi avoir deux vecteurs propres correspondant à une valeur propre simple si la valeur propre est répétée. 

En somme, les valeurs propres et les fonctions propres sont 
\begin{align*}
& \lambda_k = k^2 & & \text{avec fonctions propres} & &
\cos (k t) \quad \text{et}\quad  \sin (k t)
 & & \text{pour tous les entiers } k \geq 1 , \\
& \lambda_0 = 0 & & \text{avec fonction propre} & & x_0 = 1.
\end{align*}
\end{example}

\subsection{Orthogonalité des fonctions propres}

Une chose qui sera très utile dans la prochaine section est la propriété d'\emph{\myindex{orthogonalité}} des fonctions propres. C'est analogue aux vecteurs propres d'une matrice. Une matrice est 
\emph{symétrique\index{symmetric matrix}}
si $A = A^T$ (est égale à sa transposée).
\emph{Les vecteurs propres de deux valeurs propres distinctes d'une matrice symétrique sont orthogonaux.}
%That symmetry is required.  
%We will not prove this fact here.
L'opérateur différentiel que l'on considère agit plutôt comme  une matrice symétrique. Ainsi, on obtient le théorème suivant.  

%\medskip
%
%Suppose $\lambda_1$ and $\lambda_2$ are two distinct eigenvalues of $A$
%and $\vec{v}_1$ and $\vec{v}_2$ are the corresponding eigenvectors.  Then
%we of course have that $A \vec{v}_1 = \lambda_1 \vec{v}_1$ and
%$A \vec{v}_2 = \lambda_2 \vec{v}_2$.
%\begin{equation*}
%\langle A \vec{v}_1 , \vec{v}_2 \rangle = \lambda_1 \langle \vec{v}_1 , \vec{v}_2 \rangle
%\qquad
%\langle A \vec{v}_2 , \vec{v}_1 \rangle = \lambda_2 \langle \vec{v}_2 , \vec{v}_1 \rangle
%\end{equation*}
%
%\begin{equation*}
%\langle A \vec{v}_1 , \vec{v}_2 \rangle -
%\langle A \vec{v}_2 , \vec{v}_1 \rangle 
%=
%(\lambda_1 - \lambda_2 ) \langle \vec{v}_1 , \vec{v}_2 \rangle
%\end{equation*}
%
%\begin{equation*}
%\langle (A-A^T) \vec{v}_1 , \vec{v}_2 \rangle
%=
%(\lambda_1 - \lambda_2 ) \langle \vec{v}_1 , \vec{v}_2 \rangle
%\end{equation*}

\begin{theorem} \label{bvp:orthogonaleigen}
Soit  $x_1(t)$ et $x_2(t)$ deux fonctions propres du problème 
\eqref{bv:eq1}, \eqref{bv:eq2} ou \eqref{bv:eq3}
pour deux valeurs propres distinctes $\lambda_1$ et $\lambda_2$. Alors, elles sont
\emph{orthogonales \index{orthogonal!functions}}
dans le sens que
\begin{equation*}
\int_a^b x_1(t) x_2(t) \,dt = 0 .
\end{equation*}
\end{theorem}

La terminologie vient du fait que l'intégrale est un type de produit scalaire. On précisera ce sujet dans la prochaine section. Le théorème a une preuve  très petite, élégante et éclairante qui est présentée ici.  
D'abord, on a les deux équations suivantes : 
\begin{equation*}
x_1'' + \lambda_1 x_1 = 0
\qquad \text{et} \qquad
x_2'' + \lambda_2 x_2 = 0.
\end{equation*}
On multiplie la première par $x_2$ et la seconde par $x_1$, et l'on substitue pour obtenir 
\begin{equation*}
(\lambda_1 - \lambda_2) x_1 x_2 = x_2'' x_1 - x_2 x_1'' .
\end{equation*}
Maintenant, on intègre des deux côtés de l'équation: 
\begin{equation*}
\begin{split}
(\lambda_1 - \lambda_2) \int_a^b x_1 x_2 \,dt
& =
\int_a^b x_2'' x_1 - x_2 x_1'' \,dt \\
& =
\int_a^b \frac{d}{dt} \left( x_2' x_1 - x_2 x_1' \right) \,dt \\
& =
\Bigl[ x_2' x_1 - x_2 x_1' \Bigr]_{t=a}^b
= 0 .
\end{split}
\end{equation*}
La dernière égalité tient en raison des conditions aux limites. Par exemple, si l'on considère \eqref{bv:eq1}, on a  $x_1(a) = x_1(b) = x_2(a) = x_2(b) = 0$, et alors $x_2' x_1 - x_2 x_1'$ est nul pour $a$ et pour $b$.
Lorsque $\lambda_1 \not= \lambda_2$, le théorème suit.

\begin{exercise}[facile]
Terminez la preuve du théorème (vérifiez la dernière égalité de la preuve) pour les cas \eqref{bv:eq2} et \eqref{bv:eq3}.
\end{exercise}

La fonction $\sin (n t)$ est une fonction propre pour l'équation 
$x''+\lambda x = 0$, $x(0) = 0$, $x(\pi) = 0$. 
Donc, pour les entiers positifs $n$ et $m$:  
\begin{equation*}
\int_{0}^\pi \sin (mt) \sin (nt) \,dt = 0 ,
\quad
\text{lorsque } m \not = n.
\end{equation*}
De manière similaire,
\begin{equation*}
\int_{0}^\pi \cos (mt) \cos (nt) \,dt = 0 ,
\quad
\text{lorsque } m \not = n,
\qquad \text{et} \qquad
\int_{0}^\pi  \cos (nt) \,dt = 0 .
\end{equation*}
Et finalement, on obtient aussi
\begin{equation*}
\int_{-\pi}^\pi \sin (mt) \sin (nt) \,dt = 0 ,
\quad
\text{lorsque } m \not = n, 
\qquad \text{et} \qquad
\int_{-\pi}^\pi  \sin (nt) \,dt = 0 ,
\end{equation*}
\begin{equation*}
\int_{-\pi}^\pi \cos (mt) \cos (nt) \,dt = 0 ,
\quad
\text{lorsque } m \not = n,
\qquad \text{et} \qquad
\int_{-\pi}^\pi  \cos (nt) \,dt = 0 ,
\end{equation*}
et
\begin{equation*}
\int_{-\pi}^\pi \cos (mt) \sin (nt) \,dt = 0 
\qquad \text{(même si $m=n$).}
\end{equation*}

%\medskip
%
%The theorem is also true when different boundary conditions are applied as
%well.  For example, if we require $x'(a) = x'(b) = 0$, or
%$x(a) = x'(b) = 0$, or
%$x'(a) = x(b) = 0$.  See the proof.


%By what we have seen previously we apply the theorem to find the integrals
%\begin{equation*}
%\int_{-\pi}^\pi \sin (mt) \sin (nt) \,dt = 0 \qquad \text{and} \qquad
%\int_{-\pi}^\pi \cos (mt) \cos (nt) \,dt = 0 ,
%\end{equation*}
%when $m \not = n$, and 
%\begin{equation*}
%\int_{-\pi}^\pi \sin (mt) \cos (nt) \,dt = 0 ,
%\end{equation*}
%for all $m$ and $n$.

\subsection{Alternative de Fredholm}

On arrive maintenant à un théorème très utile des équations différentielles. Le théorème pourrait être plus général que ce qu'on présentera, mais, pour les objectifs actuels, il sera suffisant. On en donnera une version un peu plus générale plus tard.

\begin{theorem}[Alternative de Fredholm\footnote{%
Nommée d'après le mathématicien suédois
\href{https://en.wikipedia.org/wiki/Fredholm}{Erik Ivar Fredholm}
(1866-1927).}]\index{Fredholm alternative!simple case}
\label{thm:fredholmsimple}
Exactement l'une des affirmations suivantes est vraie.
Soit
\begin{equation} \label{simpfredhomeq}
x'' + \lambda x = 0, \quad x(a) = 0, \quad x(b) = 0
\end{equation}
a une solution non nulle, soit
\begin{equation} \label{simpfrednonhomeq}
x'' + \lambda x = f(t), \quad x(a) = 0, \quad x(b) = 0
\end{equation}
a une solution unique pour toutes les fonctions $f$ continues sur $[a,b]$.
\end{theorem}

Le théorème est également vrai pour les autres types de
conditions au bord que nous avons considérées.
Le théorème veut dire que si  $\lambda$ n'est pas une valeur propre, l'équation non homogène  \eqref{simpfrednonhomeq} a une solution unique pour tous les côtés droits. Autrement dit, si $\lambda$ est une valeur propre, alors  
\eqref{simpfrednonhomeq} n'a pas besoin d'avoir une solution pour tous les $f$,
et, en plus, même s'il y a une solution, la solution n'est pas unique.

Nous voulons également renforcer l'idée que les opérateurs différentiels linéaires ont beaucoup en commun avec les matrices. Il n'est donc pas surprenant qu'il existe une version de dimension finie de l'alternative de Fredholm pour les matrices  également.  On a $A$ une matrice $n \times n$. L'alternative de Fredholm dit que $(A-\lambda I) \vec{x} = \vec{0}$ a une solution non triviale, ou que $(A-\lambda I) \vec{x} = \vec{b}$ a une solution unique pour tout $\vec{b}$.

Beaucoup d'intuition de l'algèbre linéaire peut être appliquée aux opérateurs différentiels linéaires, mais il faut bien sûr être prudent. Par exemple, une différence que nous avons déjà vue est qu'en général un opérateur différentiel
aura une infinité de valeurs propres, tandis qu'une matrice n'en a qu'un nombre fini.


\subsection{Application}

Considérons une application physique. On suppose avoir un élastique très tendu qui tourne rapidement ou une corde de densité linéaire uniforme $\rho$, par exemple, en
$\unitfrac{kg}{m}$.
Mettons ce problème dans un plan $xy$ avec $x$ et $y$ en mètres. L'axe des $x$ représente la position de la corde. La corde tourne, et la vitesse angulaire est $\omega$, mesurée en $\unitfrac{radians}{s}$.
Imaginez que l'entièreté du plan $xy$ tourne à une vitesse angulaire  $\omega$.
De cette manière, la corde reste dans le plan $xy$, et $y$ mesure sa déviation par rapport à la position d'équilibre, $y=0$, sur l'axe des $x$.
Ainsi, le graphe de $y$ donne la forme de la corde.
On considère une corde idéale qui n'a pas de volume, juste une courbe mathématique. On suppose que la tension de la corde est une constante $T$ en newtons.
%If we take a small segment and we look at the tension at the endpoints, we
%see that this force is tangential and we will assume that the magnitude is
%the same at both end points.  Hence the magnitude
%is constant everywhere and we will
%call its magnitude $T$.
En supposant que la déviation est petite, nous pouvons utiliser la deuxième loi de Newton (nous n'en faisons pas la démonstration) pour obtenir l'équation suivante: 
\begin{equation*}
T y'' + \rho \omega^2 y = 0 .
\end{equation*}
Pour vérifier les unités, notez que les unités de $y''$ sont $\unitfrac{m}{m^2}$, lorsqu'on dérive par rapport à $x$.

La corde  est de longueur $L$ (en mètres) et est fixée aux deux extrémités. Alors, $y(0) = 0$ et $y(L) = 0$.  Voir la \figureref{bvp:whirstringfig}.

\begin{myfig}
\capstart
\inputpdft{bvp-whirstring}
\caption{Corde tournante.\label{bvp:whirstringfig}}
\end{myfig}

On réécrit l'équation: 
$y'' + \frac{\rho \omega^2}{T} y = 0$.
La configuration est semblable à l'\exampleref{bvp:eig1ex}, sauf que  
la longueur de l'intervalle est $L$ plutôt que $\pi$. On cherche les valeurs propres de  $y'' + \lambda y = 0, y(0) = 0, y(L) = 0$, où
$\lambda = \frac{\rho \omega^2}{T}$.  Comme précédemment, il n'y a pas de valeur propre négative. Avec $\lambda > 0$,
la solution générale à cette équation est $y = A \cos (  \sqrt{\lambda} \,x ) + B
\sin ( \sqrt{\lambda} \,x )$.  La condition $y(0) = 0$ implique que $A = 0$ comme précédemment. La condition $y(L) = 0$ implique que
$\sin ( \sqrt{\lambda} \, L) = 0$, et ainsi
$\sqrt{\lambda} \, L = k \pi$  pour un certain entier $k > 0$, alors
\begin{equation*}
\frac{\rho \omega^2}{T} = \lambda = \frac{k^2 \pi^2}{L^2} .
\end{equation*}

Qu'est-ce que cela dit sur la forme de la corde? Cela dit que, pour tous les paramètres $\rho$, $\omega$, $T$ qui ne satisfont pas à cette condition, la corde demeure dans sa position d'équilibre, $y=0$.  Mais lorsque 
$\frac{\rho \omega^2}{T} = \frac{k^2 \pi^2}{L^2}$, alors la corde \myquote{sortira} d'une distance $B$. On ne peut pas calculer $B$ avec les informations que l'on a. 

On suppose que  $\rho$ et $T$ sont fixes et l'on fait varier $\omega$.
Pour la plupart des valeurs de $\omega$, la corde est à l'état d'équilibre. Lorsque la vitesse angulaire  $\omega$ atteint une valeur $\omega = \frac{k \pi \sqrt{T}}{L\sqrt{\rho}}$, alors la corde sort et a la forme d'une vague sinusoïdale croisant l'axe des 
$x$ $k-1$ fois entre les extrémités. Par exemple, à $k=1$, la corde ne croise pas l'axe des $x$, et la forme ressemble à la \figurevref{bvp:whirstringfig} plus haut.
D'un autre côté, lorsque  $k=3$, la corde croise l'axe des $x$ deux fois. On le voit dans la  \figurevref{bvp:whirstring2fig}.
Lorsque $\omega$ change encore, la corde retourne à sa position d'équilibre. Plus la vitesse angulaire est élevée, plus la corde croisera l'axe des  $x$ lorsqu'elle sortira.

\begin{myfig}
\capstart
\inputpdft{bvp-whirstring2}
\caption{Corde tournante à la troisième valeur propre ($k=3$).\label{bvp:whirstring2fig}}
\end{myfig}

Pour un autre exemple, si l'on a une corde à sauter qui tourne (alors $k=1$ lorsqu'elle sort \myquote{complètement}) et que l'on tire sur les extrémités pour augmenter la tension, alors la vitesse augmente pour que la corde reste \myquote{sortie}.
 

\subsection{Exercices}

Astuce pour les exercices suivants:  Notez que lorsque  $\lambda > 0$, alors
$\cos \bigl( \sqrt{\lambda}\, (t - a) \bigr)$
et $\sin  \bigl( \sqrt{\lambda}\, (t - a) \bigr)$
sont aussi des solutions de l'équation homogène. 

\begin{exercise}
Calculez toutes les valeurs propres et toutes les fonctions propres de 
$x'' + \lambda x = 0, ~ x(a) = 0, ~ x(b) = 0$ (supposez que $a < b$).
\end{exercise}

\begin{exercise}
Calculez toutes les valeurs propres et toutes les fonctions propres de 
$x'' + \lambda x = 0, ~ x'(a) = 0, ~ x'(b) = 0$ (supposez que $a < b$).
\end{exercise}

\begin{exercise}
Calculez toutes les valeurs propres et toutes les fonctions propres de 
$x'' + \lambda x = 0, ~ x'(a) = 0, ~ x(b) = 0$ (supposez que $a < b$).
\end{exercise}

\begin{exercise}
Calculez toutes les valeurs propres et toutes les fonctions propres de 
$x'' + \lambda x = 0, ~ x(a) = x(b), ~ x'(a) = x'(b)$ (supposez que $a < b$).
\end{exercise}

\begin{exercise}
On ne considérera pas le cas  $\lambda < 0$ pour la valeur à l'extrêmité du problème f
$x'' + \lambda x = 0, ~ x(-\pi) = x(\pi), ~ x'(-\pi) = x'(\pi)$.
Terminez le calcul et montrez qu'il n'y a pas de valeur propre négative. 
\end{exercise}

\setcounter{exercise}{100}

\begin{exercise}
Considérez une corde, de longueur 2, de densité linéaire de 0,1 et de tension 3,  qui tourne. Trouvez la plus petite vitesse angulaire permettant à la corde de sortir. 
\end{exercise}
\exsol{%
$\omega = \pi \sqrt{\frac{15}{2}}$
}

\begin{exercise}
Supposez que $x'' + \lambda x = 0$ et $x(0)=1$, $x(1) = 1$.
Trouvez tous les $\lambda$ pour lesquels il y a plus d'une solution. Trouvez aussi les solutions correspondantes (seulement pour les valeurs propres). 
\end{exercise}
\exsol{%
$\lambda_k = 4 k^2 \pi^2$ pour $k = 1,2,3,\ldots$
\quad
$x_k =  \cos (2k\pi t) + B \sin (2k\pi t)$ \quad (pour tout $B$)
}

\begin{exercise}
Supposez que $x'' + x = 0$ et $x(0)=0$, $x'(\pi) = 1$.
Trouvez toute(s) les solution(s) si elles existent.  
\end{exercise}
\exsol{%
$x(t) = - \sin(t)$
}

\begin{exercise}
Considerez
$x' + \lambda x = 0$ et $x(0)=0$, $x(1) = 0$.  Pourquoi il n'y a pas de valeurs propres? Pourquoi toutes les équation de premier ordre avec deux conditions au bord comme ci-haut n'ont pas de valeurs propres? 
\end{exercise}
\exsol{%
La solution générale est $x = C e^{-\lambda t}$.  Depuis que $x(0) = 0$ alors $C=0$, et alors $x(t) = 0$.
Ainsi, la solution est toujours identiquement nulle. Une condition est toujours suffisante pour garantir une solution unique pour une équation de premier ordre. 
}

\begin{exercise}[défi]
Supposez que $x''' + \lambda x = 0$ et $x(0)=0$, $x'(0) = 0$, $x(1) = 0$.
Supposez que  $\lambda > 0$.  Trouvez une équation à laquelle toutes les valeurs propres satisferont. 
Astuce : Notez que $-\sqrt[3]{\lambda}$ est une racine de $r^3+\lambda = 0$.
\end{exercise}
\exsol{%
$\frac{\sqrt{3}}{3} e^{\frac{-3}{2}\sqrt[3]{\lambda}}
- \frac{\sqrt{3}}{3} \cos \bigl( \frac{\sqrt{3}\, \sqrt[3]{\lambda}}{2} \bigr)
+ \sin \bigl( \frac{\sqrt{3}\, \sqrt[3]{\lambda}}{2}\bigr) = 0$
}

%%%%%%%%%%%%%%%%%%%%%%%%%%%%%%%%%%%%%%%%%%%%%%%%%%%%%%%%%%%%%%%%%%%%%%%%%%%%%%

\sectionnewpage
\section{Séries trigonométriques} \label{ts:section}

\subsection{Fonctions périodiques et motivations}

Afin de motiver l'étude des séries de Fourier, considérons le problème suivant:
\begin{equation} \label{ts:deq}
x'' + \omega_0^2 x = f(t)
\end{equation}
pour une certaine fonction périodique $f(t)$.
On a déjà résolu
\begin{equation} \label{ts:deqcos}
x'' + \omega_0^2 x = F_0 \cos ( \omega t) .
\end{equation}
Une manière de résoudre \eqref{ts:deq} est de décomposer $f(t)$ comme une somme de cosinus (et de sinus) et de résoudre plusieurs problèmes de la forme  \eqref{ts:deqcos}.  On utilise ensuite le principe de superposition pour additionner toutes les solutions obtenues pour obtenir la solution de \eqref{ts:deq}.

Avant de procéder, parlons un peu plus en détail des fonctions périodiques. 
Une fonction est  \emph{\myindex{périodique}} avec période $P$ si
$f(t) = f(t+P)$ pour tout $t$.  On dit alors que $f(t)$ est $P$-périodique.
On note qu'une fonction $P$-périodique est aussi $2P$-périodique, $3P$-périodique
et ainsi de suite.
Par exemple, $\cos (t)$ et $\sin (t)$ sont
$2\pi$-périodiques.  
% Alors, $\cos (kt)$ et $\sin (kt)$ pour tous les entiers $k$.  
Les fonctions constantes sont des exemples extrêmes: elles sont périodiques pour n'importe quelle période (exercice). 

Normalement, on commence avec la fonction $f(t)$ définie sur un certain intervalle $[-L,L]$ et l'on veut \emph{prolonger périodiquement}\index{extend periodically}\index{periodic extension} $f(t)$ pour qu'elle devienne une fonction $2L$-périodique.  
On fait cette extension en définissant une nouvelle fonction $F(t)$
telle que, pour tout $t$ dans $[-L,L]$, $F(t) = f(t)$. Pour tout $t$ dans $[L,3L]$,
on définit $F(t) = f(t-2L)$; pour $t$ dans $[-3L,-L]$, $F(t) = f(t+2L)$, et ainsi de suite.
Pour faire ce travail, on a besoin de $f(-L) = f(L)$.
On pourrait aussi commencer avec $f$
définie seulement sur un demi-intervalle ouvert $(-L,L]$ et on définit $f(-L) = f(L)$.

\begin{example}
Définissons  $f(t) = 1-t^2$ sur $[-1,1]$.  Maintenant, on prolonge $f(t)$ périodiquement pour obtenir une fonction 2-périodique. Voir la \figureref{ts:perextofinvertedparabolafig}.
\begin{myfig}
\capstart
\diffyincludegraphics{width=3in}{width=4.5in}{ts-perextofinvertedparabola}
\caption{Prolongement 2-périodique de la fonction 
$1-t^2$.\label{ts:perextofinvertedparabolafig}}
\end{myfig}
\end{example}

On doit faire attention à bien distinguer $f(t)$ de ses différents prolongements.  Une erreur commune est de supposer que la formule pour $f(t)$ fonctionne toujours pour ses prolongements. Il faut faire preuve de prudence, particulièrement lorsque $f(t)$ est périodique, mais avec une période différente.

\begin{exercise}
Définissez $f(t) = \cos t$ sur $[\nicefrac{-\pi}{2},\nicefrac{\pi}{2}]$.  Prenez l'extension $\pi$-périodique et esquissez le graphe. Comparez avec le graphe de $\cos t$.
\end{exercise}

\subsection{Produit scalaire et décomposition de vecteurs propres }

Comme on l'a remarqué précédemment, lorsque $A$ est une \emph{\myindex{matrice symétrique}},
c'est-à-dire lorsque $A^T = A$, les vecteurs propres de $A$ sont orthogonaux. Ici, le mot 
\emph{orthogonal}\index{orthogonal!vectors} signifie que si $\vec{v}$ et $\vec{w}$ sont deux vecteurs propres de $A$ pour des valeurs propres distinctes, alors $\langle \vec{v} , \vec{w} \rangle = 0$.
Dans ce cas, le produit scalaire  $\langle \vec{v} , \vec{w} \rangle$
est le \emph{\myindex{produit scalaire à dimensions finies}}, que l'on peut calculer comme $\vec{v}^T\vec{w}$.

Pour décomposer un vecteur $\vec{v}$ en termes de vecteurs mutuellement orthogonaux  $\vec{w}_1$ et $\vec{w}_2$, on écrit
\begin{equation*}
\vec{v} = a_1 \vec{w}_1  + a_2 \vec{w}_2 .
\end{equation*}
On trouve la formule pour $a_1$ et $a_2$. D'abord, on calcule 
\begin{equation*}
\langle \vec{v} , \vec{w_1} \rangle
=
\langle a_1 \vec{w}_1  + a_2 \vec{w}_2 , \vec{w_1} \rangle
=
a_1 \langle \vec{w}_1 , \vec{w_1} \rangle
+
a_2 \underbrace{\langle \vec{w}_2 , \vec{w_1} \rangle}_{=~0}
=
a_1 \langle \vec{w}_1 , \vec{w_1} \rangle .
\end{equation*}
Ensuite,
\begin{equation*}
a_1 = 
\frac{\langle \vec{v} , \vec{w_1} \rangle}{
\langle \vec{w}_1 , \vec{w_1} \rangle} .
\end{equation*}
De façon semblable,
\begin{equation*}
a_2 = 
\frac{\langle \vec{v} , \vec{w_2} \rangle}{
\langle \vec{w}_2 , \vec{w_2} \rangle} .
\end{equation*}
(Vous avez possiblement vu cette formule dans un cours d'algèbre linéaire.)

\begin{example}
Écrivons 
$\vec{v} = \left[ \begin{smallmatrix} 2 \\ 3 \end{smallmatrix} \right]$
comme une combinaison linéaire de 
$\vec{w_1} = \left[ \begin{smallmatrix} 1 \\ -1 \end{smallmatrix} \right]$
et de
$\vec{w_2} = \left[ \begin{smallmatrix} 1 \\ 1 \end{smallmatrix} \right]$.

D'abord, on note que $\vec{w}_1$ et $\vec{w}_2$ sont orthogonaux, puisque 
 $\langle \vec{w}_1 , \vec{w}_2 \rangle = 1(1) + (-1)1 = 0$.
Alors,
\begin{align*}
& a_1 = 
\frac{\langle \vec{v} , \vec{w_1} \rangle}{
\langle \vec{w}_1 , \vec{w_1} \rangle}
=
\frac{2(1) + 3(-1)}{1(1) + (-1)(-1)} = \frac{-1}{2} ,
\\
& a_2 = 
\frac{\langle \vec{v} , \vec{w_2} \rangle}{
\langle \vec{w}_2 , \vec{w_2} \rangle}
=
\frac{2 + 3}{1 + 1} = \frac{5}{2} .
\end{align*}
Ainsi,
\begin{equation*}
\begin{bmatrix} 2 \\ 3 \end{bmatrix}
=
\frac{-1}{2}
\begin{bmatrix} 1 \\ -1 \end{bmatrix}
+
\frac{5}{2}
\begin{bmatrix} 1 \\ 1 \end{bmatrix} .
\end{equation*}
\end{example}

\subsection{Séries trigonométriques}

Plutôt que de décomposer un vecteur en termes de vecteurs propres d'une matrice, on va décomposer une fonction en termes de fonctions propres associées à un problème de valeurs propres. Le problème de valeurs propres qu'on utilise pour les séries de Fourier est 
\begin{equation*}
x'' + \lambda x = 0, \quad x(-\pi) = x(\pi), \quad x'(-\pi) = x'(\pi) .
\end{equation*}
Nous avons calculé les fonctions propres dans l'exemple \ref{bvp-periodic:example} : $1$, $\cos (k t)$ et
$\sin (k t)$.  Ainsi, on veut trouver une représentation d'une fonction 
$2\pi$-périodique $f(t)$ de la forme suivante: 
\begin{equation*}
\mybxbg{~~
f(t) = \frac{a_0}{2} +
\sum_{n=1}^\infty a_n \cos (n t) + b_n \sin (n t) .
~~}
\end{equation*}
Cette série s'appelle une \emph{\myindex{série de Fourier}}\footnote{%
Nommée d'après le mathématicien français
\href{https://en.wikipedia.org/wiki/Joseph_Fourier}{Jean Baptiste Joseph Fourier}
(1768-1830).} ou
\emph{\myindex{série trigonométrique}} pour $f(t)$.
Par convention, on écrit le coefficient de la fonction propre 1 comme $\frac{a_0}{2}$. On peut y penser comme à $1 = \cos (0t)$.

Comme pour les matrices, on veut trouver une  \emph{\myindex{projection}}
de $f(t)$ sur un sous-espace donné par les fonctions propres. Alors, on veut définir le \emph{\myindex{produit scalaire d'une fonction}}. Par exemple, pour trouver $a_n$, on veut calculer $\langle \, f(t), \, \cos (nt) \, \rangle$.
On définit le produit scalaire comme
\begin{equation*}
\langle \, f(t), \, g(t) \, \rangle \overset{\text{déf}}{=}
\int_{-\pi}^\pi f(t) \, g(t) \, dt .
\end{equation*}
Avec cette définition du produit scalaire, on voit dans les sections précédentes que les fonctions propres  $\cos (kt)$
(incluant la fonction propre constante) et
$\sin (kt)$ sont \emph{orthogonales\index{orthogonal!functions}} dans le sens que
\begin{align*}
\langle \, \cos (mt), \, \cos (nt) \, \rangle = 0 & \qquad \text{pour } m \not= n , \\
\langle \, \sin (mt), \, \sin (nt) \, \rangle = 0 & \qquad \text{pour } m \not= n , \\
\langle \, \sin (mt), \, \cos (nt) \, \rangle = 0 & \qquad \text{pour tout } m \text{ et } n .
\end{align*}
Pour $n=1,2,3,\ldots$,
on a
\begin{align*}
\langle \, \cos (nt), \, \cos (nt) \, \rangle &=
\int_{-\pi}^\pi \cos(nt)\cos(nt) \, dt
=
\pi,
\\
\langle \, \sin (nt), \, \sin (nt) \, \rangle &=
\int_{-\pi}^\pi \sin(nt)\sin(nt) \, dt
=
\pi,
\end{align*}
par du calcul élémentaire. Pour la constante, on obtient
\begin{equation*}
\langle \, 1, \, 1 \, \rangle
=
\int_{-\pi}^\pi 1 \cdot 1 \, dt
 = 2\pi.
\end{equation*}
Les coefficients sont donnés par
\begin{equation*}
\mybxbg{~~
\begin{aligned}
& a_n =
\frac{\langle \, f(t), \, \cos (nt) \, \rangle}{\langle \, \cos (nt) \, , \,
\cos (nt) \, \rangle}
= 
\frac{1}{\pi} \int_{-\pi}^\pi f(t) \cos (nt) \, dt , \\
& b_n =
\frac{\langle \, f(t), \, \sin (nt) \, \rangle}{\langle \, \sin (nt), \,
\sin (nt) \, \rangle}
= 
\frac{1}{\pi} \int_{-\pi}^\pi f(t) \sin (nt) \, dt .
\end{aligned}
~~}
\end{equation*}
On compare ces expressions avec les exemples à dimensions finies. 
Pour $a_0$, on obtient une formule similaire :
\begin{equation*}
\mybxbg{~~
a_0 = 2
\frac{\langle \, f(t), \, 1 \, \rangle}{\langle \, 1, \,
1 \, \rangle}
=
\frac{1}{\pi} \int_{-\pi}^\pi f(t) \, dt .
~~}
\end{equation*}

Regardons la formule utilisée pour les propriétés d'orthogonalité. Supposons donc que:  
\begin{equation*}
f(t) = \frac{a_0}{2} + \sum_{n=1}^\infty a_n \cos (n t) + b_n
\sin (n t) .
\end{equation*}
Alors, pour $m \geq 1$, on a
\begin{equation*}
\begin{split}
\langle \, f(t),\,\cos (mt) \, \rangle
& =
\Bigl\langle \, \frac{a_0}{2} + \sum_{n=1}^\infty a_n \cos (n t) + b_n
\sin (n t),\, \cos (mt) \, \Bigr\rangle \\
& =
\frac{a_0}{2}
\langle \, 1, \, \cos (mt) \, \rangle
+ \sum_{n=1}^\infty
a_n \langle \, \cos (nt), \, \cos (mt) \, \rangle +
b_n \langle \, \sin (n t), \, \cos (mt) \, \rangle \\
& =
a_m \langle \, \cos (mt), \, \cos (mt) \, \rangle .
\end{split}
\end{equation*}
Ainsi,
$a_m =
\frac{\langle \, f(t), \, \cos (mt) \, \rangle}{\langle \, \cos (mt), \,
\cos (mt) \, \rangle}$.

\begin{exercise}
Faites les calculs pour $a_0$ et pour $b_m$.
\end{exercise}

\begin{example}
Prenons la fonction
\begin{equation*}
f(t) = t
\end{equation*}
pour $t$ dans $(-\pi,\pi]$. Nous allons calculer la série de Fourier du prolongement périodique de $f(t)$.  Le terme \emph{\myindex{dents de scie}} est utilisé pour désigner les fonctions de cette forme.  Le graphe de $f(t)$ se trouve à la \figureref{ts:sawtoothfig}.

\begin{myfig}
\capstart
\diffyincludegraphics{width=3in}{width=4.5in}{ts-sawtooth}
\caption{Le graphe d'une fonction en dents de scie.\label{ts:sawtoothfig}}
\end{myfig}
Calculons les coefficients. On commence avec  $a_0$: 
\begin{equation*}
a_0 = \frac{1}{\pi} \int_{-\pi}^\pi t \,dt = 0 .
\end{equation*}
Nous nous servirons souvent du résultat suivant: sur un intervalle de la forme $[-L,L]$,  l'intégrale d'une fonction impaire est nulle. On rappelle qu'une 
\emph{\myindex{fonction impaire}} est une fonction 
 $\varphi(t)$ telle que $\varphi(-t) = -\varphi(t)$.  Par exemple, les fonctions $t$, $\sin t$ ou (celle qui est importante présentement)
$t \cos (nt)$ sont toutes des fonctions impaires.  Ainsi,
\begin{equation*}
a_n = \frac{1}{\pi} \int_{-\pi}^\pi t \cos (nt) \,dt = 0 .
\end{equation*}
Calculons maintenant $b_n$.  Deuxième résultat utile: sur un intervalle de la forme $[-L,L]$, l'intégrale d'une fonction paire est égale à deux fois l'intégrale de cette même fonction sur $[0,L]$. On rappelle qu'une  \emph{\myindex{fonction paire}}
est une fonction $\varphi(t)$ telle que  $\varphi(-t) = \varphi(t)$.  Par exemple,
$t \sin (nt)$ est paire, et donc: 
\begin{equation*}
\begin{split}
b_n & = \frac{1}{\pi} \int_{-\pi}^\pi t \sin (nt) \,dt \\
& = \frac{2}{\pi} \int_{0}^\pi t \sin (nt) \,dt \\
& = \frac{2}{\pi} \left(
\left[ \frac{-t \cos (nt)}{n} \right]_{t=0}^{\pi}
+
\frac{1}{n}
\int_{0}^\pi \cos (nt) \,dt
\right)
\\
& = \frac{2}{\pi} \left(
\frac{-\pi \cos (n\pi)}{n}
+
0
\right) \\
& =  \frac{-2 \cos (n\pi)}{n}
=  \frac{2 \,{(-1)}^{n+1}}{n} .
\end{split}
\end{equation*}
On a utilisé ici le fait suivant :  
\begin{equation*}
\cos (n\pi) = {(-1)}^n =
\begin{cases}
1 & \text{si } n \text{ est pair} , \\
-1 & \text{si } n \text{ est impair} .
\end{cases}
\end{equation*}
La série est
\begin{equation*}
\sum_{n=1}^\infty
\frac{2 \,{(-1)}^{n+1}}{n} \,
\sin (n t) .
\end{equation*}

Écrivons les trois premières harmoniques de la série pour $f(t)$: 
\begin{equation*}
2 \, \sin (t)
- \sin (2t)
+\frac{2}{3} \sin (3t)
+ \cdots
\end{equation*}
Le graphe des trois premiers termes de la série et celui des vingt premiers se trouvent à la~\figureref{ts:sawtoothfsfig}.

\begin{myfig}
\capstart
%original files ts-sawtooth-fs3 ts-sawtooth-fs20
\diffyincludegraphics{width=6.24in}{width=9in}{ts-sawtooth-fs3-fs20}
\caption{Trois premières harmoniques (à gauche) et vingt premières (à droite) d'une fonction en dents de scie.\label{ts:sawtoothfsfig}}
\end{myfig}
\end{example}

\begin{example}
Prenons la fonction
\begin{equation*}
f(t) =
\begin{cases}
0 & \text{si } \;{-\pi} < t \leq 0 , \\
\pi & \text{si } \;\phantom{-}0 < t \leq \pi .
\end{cases}
\end{equation*}
\nopagebreak[4]%
Prolongeons $f(t)$ périodiquement et écrivons-la comme une série de Fourier.  Cette fonction ou une de ses variantes apparaissent dans plusieurs applications, et on les appelle des 
\emph{\myindex{vagues carrées}}. Le graphe du prolongement de $f(t)$ est montré à la~\figureref{ts:squarewavefig}.

\begin{myfig}
\capstart
\diffyincludegraphics{width=3in}{width=4.5in}{ts-squarewave}
\caption{Le graphe d'une vague carrée.\label{ts:squarewavefig}}
\end{myfig}

Maintenant, on calcule les coefficients. On commence avec $a_0$ :
\begin{equation*}
a_0 = \frac{1}{\pi} \int_{-\pi}^\pi f(t) \,dt
= \frac{1}{\pi} \int_{0}^\pi \pi \,dt = \pi .
\end{equation*}
Ensuite,
\begin{equation*}
a_n = \frac{1}{\pi} \int_{-\pi}^\pi f(t) \cos (nt) \,dt 
= \frac{1}{\pi} \int_{0}^\pi \pi \cos (nt) \,dt = 0 .
\end{equation*}
Et finalement,
\begin{equation*}
\begin{split}
b_n & = \frac{1}{\pi} \int_{-\pi}^\pi f(t) \sin (nt) \,dt \\
& = \frac{1}{\pi} \int_{0}^\pi \pi \sin (nt) \,dt \\
& = \left[ \frac{- \cos (nt)}{n} \right]_{t=0}^\pi \\
& = \frac{1 - \cos (\pi n)}{n}
= \frac{1 - {(-1)}^n}{n}
=
\begin{cases}
\frac{2}{n} & \text{si } n \text{ est impair} , \\
0 & \text{si } n \text{ est pair} .
\end{cases}
\end{split}
\end{equation*}
La série de Fourier est
\begin{equation*}
\frac{\pi}{2} +  \sum_{\substack{n=1\\n \text{ impair}}}^\infty
\frac{2}{n} 
\sin (n t)
=
\frac{\pi}{2} + \sum_{k=1}^\infty
\frac{2}{2k-1} 
\sin \bigl( (2k-1)\, t \bigr) .
\end{equation*}

Voici les trois premières harmoniques de la série pour  $f(t)$ :
\begin{equation*}
\frac{\pi}{2}
+
2 \, \sin (t)
+
\frac{2}{3}  \sin (3t)
+ \cdots
\end{equation*}
Le graphe des trois premières harmoniques de la série (et aussi des vingt premières) se trouve à la~\figureref{ts:squarewavefsfig}.

\begin{myfig}
\capstart
%original files ts-squarewave-fs3 ts-squarewave-fs20
\diffyincludegraphics{width=6.24in}{width=9in}{ts-squarewave-fs3-fs20}
\caption{Trois premières harmoniques (à gauche) et vingt premières harmoniques (à droite) d'une vague carrée.\label{ts:squarewavefsfig}}
\end{myfig}
\end{example}

On a jusqu'ici évité la question de la convergence. Par exemple, si  $f(t)$ est une vague carrée, l'égalité 
\begin{equation*}
f(t) = 
\frac{\pi}{2} + \sum_{k=1}^\infty
\frac{2}{2k-1} 
\sin \bigl( (2k-1)\, t \bigr) 
\end{equation*}
tient seulement pour les valeurs de $t$ où $f(t)$ est continue. On n'obtient pas l'égalité pour $t=-\pi,0,\pi$ et pour toutes les autres discontinuités de $f(t)$. Ce n'est pas difficile de voir que lorsque $t$ est un entier muliple de 
$\pi$ (ce qui inclut la discontinuité), alors
\begin{equation*}
\frac{\pi}{2} + \sum_{k=1}^\infty
\frac{2}{2k-1} 
\sin \bigl( (2k-1)\, t \bigr) = \frac{\pi}{2} .
\end{equation*}
Redéfinissons $f(t)$ sur $[-\pi,\pi]$ comme suit,
\begin{equation*}
f(t) =
\begin{cases}
0 & \text{si } \; {-\pi} < t < 0 , \\
\pi & \text{si } \; \phantom{-}0 < t < \pi , \\
\nicefrac{\pi}{2} & \text{si } \; \phantom{-}t = -\pi, 
t = 0\text{ ou }
t = \pi,
\end{cases}
\end{equation*}
et prolongeons périodiquement.  Maintenant, la série est égale au prolongement de $f(t)$ partout, y compris aux points de discontinuité. Généralement, on ne s'en fera pas avec le changement de valeur des fonctions à certains points (en nombre fini). 

On discutera davantage de la convergence dans la prochaine section. Cependant, mentionnons brièvement un effet de la discontinuité ici. On zoome dans le voisinage de la discontinuité d'une vague carrée. Ensuite, on trace le graphe des 100 premières harmoniques, comme à 
la~\figureref{ts:squarewavegibbsfig}.  Tandis que la
série est une très bonne approximation loin des discontinuités, l'erreur
(le dépassement) dans le voisinage de la discontinuité à $t=\pi$ ne semble pas devenir plus petite. Ce comportement est connu sous le nom de \emph{\myindex{phénomène de Gibbs}}.
La région où l'erreur est large ne diminue pas, même lorsqu'on prend plus de termes dans la série. 

\begin{myfig}
\capstart
\diffyincludegraphics{width=3in}{width=4.5in}{ts-squarewave-gibbs}
\caption{Phénomène de Gibbs en action.\label{ts:squarewavegibbsfig}}
\end{myfig}

On peut penser à une fonction périodique comme à un \myquote{signal} obtenu par la superposition de plusieurs signaux de fréquence pure. Par exemple, on peut penser à la vague carrée comme un à ton d'une certaine fréquence de base. Cette fréquence est nommée la 
\emph{\myindex{fréquence fondamentale}}.
La vague carrée sera la superposition de plusieurs tons purs de fréquences qui sont des multiples de la fréquence fondamentale. En musique, les plus hautes fréquences s'appellent les \emph{\myindex{harmoniques}}.
L'ensemble des fréquences apparaissant s'appelle le 
\emph{\myindex{spectre}} du signal. Par contre, une onde sinusoïdale est un ton pur (pas d'harmonique). La manière la plus simple de créer un son en utilisant un ordinateur est avec des vagues carrées, et le son est vraiment différent du ton pur. Si vous avez déjà joué à des jeux vidéo des années 1980, vous avez entendu des sons de vagues carrées. 


\subsection{Exercices}

\begin{exercise}
Supposez que $f(t)$ est défini sur $[-\pi,\pi]$ comme $\sin (5t) + \cos (3t)$.  Prolongez périodiquement et calculez la série de Fourier pour $f(t)$.
\end{exercise}

\begin{exercise}
Supposez que $f(t)$ est défini sur $[-\pi,\pi]$ comme $\lvert t \rvert$.
  Prolongez périodiquement et calculez la série de Fourier pour $f(t)$.
\end{exercise}

\begin{exercise}
Supposez que $f(t)$ est défini sur $[-\pi,\pi]$ comme $\lvert t \rvert^3$.
Prolongez périodiquement et calculez la série de Fourier pour $f(t)$.
\end{exercise}

\begin{exercise}
Supposez que $f(t)$ est défini sur $(-\pi,\pi]$ comme
\begin{equation*}
f(t) =
\begin{cases}
-1 & \text{si } \; {-\pi} < t \leq 0 , \\
1 & \text{si } \; \phantom{-}0 < t \leq \pi .
\end{cases}
\end{equation*}
Prolongez périodiquement et calculez la série de Fourier pour $f(t)$.
\end{exercise}

\begin{exercise}
Supposez que $f(t)$ est défini sur  $(-\pi,\pi]$ comme $t^3$.
Prolongez périodiquement et calculez la série de Fourier pour $f(t)$.
\end{exercise}

\begin{exercise}
Supposez que $f(t)$ est défini sur $[-\pi,\pi]$ comme $t^2$.
Prolongez périodiquement et calculez la série de Fourier pour $f(t)$.
\end{exercise}

Il y a une autre forme de la série de Fourier qui utilise l'exponentielle complexe 
$e^{int}$ pour $n=\ldots,-2,-1,0,1,2,\ldots$ plutôt que
$\cos(nt)$ et $\sin(nt)$ pour $n$ positif.  Cette forme pourrait être plus facile à travailler à certains moments. Elle est certainement plus compacte dans son écriture et il y a seulement une formule pour les coefficients. Par contre, les coefficients sont des nombres complexes. 

\begin{exercise}
\begin{samepage}
Soit 
\begin{equation*}
f(t) = \frac{a_0}{2} + \sum_{n=1}^\infty a_n \cos (n t)
+ b_n \sin (n t) .
\end{equation*}
Utilisez la formule d'Euler $e^{i\theta} = \cos (\theta) + i \sin (\theta)$ pour montrer qu'il existe des nombres complexes 
 $c_m$ tels que
\begin{equation*}
f(t) = 
\sum_{m=-\infty}^\infty c_m e^{imt} .
\end{equation*}
Notez que la somme s'étend maintenant sur tous les entiers, y compris les nombres négatifs.
Ne vous inquiétez pas de la convergence dans ce calcul.
Astuce: Il peut être préférable de partir de la forme exponentielle complexe et d'écrire
la série comme 
\begin{equation*}
c_0 + \sum_{m=1}^\infty \Bigl( c_m e^{imt} + c_{-m} e^{-imt}  \Bigr).
\end{equation*}
\end{samepage}
\end{exercise}

\setcounter{exercise}{100}

\begin{exercise}
Supposez que $f(t)$ est défini sur $[-\pi,\pi]$ comme $f(t) = \sin(t)$. Prolongez périodiquement et calculez la série de Fourier.
\end{exercise}
\exsol{%
$\sin(t)$
}

\begin{exercise}
Supposez que $f(t)$ est défini sur $(-\pi,\pi]$ comme $f(t) = \sin(\pi t)$.  Prolongez périodiquement et calculez la série de Fourier.
\end{exercise}
\exsol{%
$\sum\limits_{n=1}^\infty
\frac{(\pi-n) \sin( \pi n+{\pi}^{2})
+(\pi+n)\sin(\pi n-{\pi}^{2}) }{\pi {n}^{2}-{\pi}^{3}}
\sin(nt)$
%$a_0 = \frac{1}{\pi} \int_{-\pi}^\pi \sin(\pi t) \, dt$
%\\
%$a_n =
%\frac{1}{\pi} \int_{-\pi}^\pi \sin(\pi t) \cos (nt) \, dt$
%\\
%$b_n =
%\frac{1}{\pi} \int_{-\pi}^\pi f(t) \sin (nt) \, dt$
}

\begin{exercise}
Supposez que $f(t)$ est défini sur $(-\pi,\pi]$ comme $f(t) = \sin^2(t)$.
Prolongez périodiquement et calculez la série de Fourier.
\end{exercise}
\exsol{%
$\frac{1}{2}-\frac{1}{2}\cos(2t)$
}

\begin{exercise}
Supposez que $f(t)$ est défini sur$(-\pi,\pi]$ comme $f(t) = t^4$.
Prolongez périodiquement et calculez la série de Fourier.
\end{exercise}
\exsol{%
$\frac{\pi^4}{5} + \sum\limits_{n=1}^\infty
\frac{{(-1)}^{n} (8{\pi}^{2}{n}^{2}-48) }{{n}^{4}}
\cos(nt)$
}

%%%%%%%%%%%%%%%%%%%%%%%%%%%%%%%%%%%%%%%%%%%%%%%%%%%%%%%%%%%%%%%%%%%%%%%%%%%%%%

\sectionnewpage
\section{Plus sur les séries de Fourier}
\label{moreonfourier:section}
%
%\sectionnotes{2 lectures\EPref{, \S9.2--\S9.3 dans \cite{EP}}\BDref{,
%\S10.3 dans \cite{BD}}}

%Before reading the lecture, it may be good to first try
%Project IV (Fourier series)\index{IODE software!Project IV} from the
%IODE website: \url{http://www.math.uiuc.edu/iode/}.  After reading the
%lecture it may be good to continue with 
%Project V (Fourier series again)\index{IODE software!Project V}.

\subsection{Fonctions $2L$-périodiques}

On a calculé les séries de Fourier pour une fonction $2\pi$-périodique, mais qu'en est-il des fonctions ayant des périodes différentes? En fait, le calcul est un simple cas de changement de variable. On doit simplement  redimensionner l'axe indépendant. On suppose avoir une fonction  $2L$-périodique $f(t)$.  Alors, $L$ est appelé la  \emph{\myindex{demie-période}}.  Soit $s = \frac{\pi}{L}  t$.
Alors la fonction
\begin{equation*}
g(s) = f\left(\frac{L}{\pi} s \right)
\end{equation*}
est $2\pi$-périodique.  On doit aussi redimensionner tous les sinus et les cosinus. Dans les séries, on utilise $\frac{\pi}{L} t$ comme une variable. Ce qu'on veut écrire est
\begin{equation*}
\mybxbg{~~
f(t) = 
\frac{a_0}{2} +
\sum_{n=1}^\infty a_n \cos \left( \frac{n \pi}{L} t \right)
+ b_n \sin \left(\frac{n \pi}{L} t \right) .
~~}
\end{equation*}
Si l'on change les variables à $s$, on voit que 
\begin{equation*}
g(s) = 
\frac{a_0}{2} +
\sum_{n=1}^\infty a_n \cos (n s)
+ b_n \sin (n s) .
\end{equation*}
On calcule $a_n$ et $b_n$ comme avant.  Ensuite, on écrit les intégrales, on change les variables de $s$ à $t$, tel que $ds = \frac{\pi}{L} \, dt$.
\begin{equation*}
\mybxbg{~~
\begin{aligned}
& a_0 =
\frac{1}{\pi}
\int_{-\pi}^\pi
g(s) \, ds
=
\frac{1}{L}
\int_{-L}^L
f(t) \, dt , \\
& a_n =
\frac{1}{\pi}
\int_{-\pi}^\pi
g(s) \, \cos (n s) \, ds
=
\frac{1}{L}
\int_{-L}^L
f(t) \, \cos \left( \frac{n \pi}{L} t \right) \, dt , \\
& b_n =
\frac{1}{\pi}
\int_{-\pi}^\pi
g(s) \, \sin (n s) \, ds
=
\frac{1}{L}
\int_{-L}^L
f(t) \, \sin \left( \frac{n \pi}{L} t \right) \, dt .
\end{aligned}
~~}
\end{equation*}

Les demies périodes de $\pi$ ou 1 sont les plus communes parce qu'elles ont une formule simple. On remarque qu'on n'a pas fait de nouvelles mathématiques, on change simplement les variables. Si l'on comprend les séries de Fourier pour les fonctions  $2\pi$-périodiques, on comprend les unités différentes dans le temps. Tout ce qu'on fait c'est de bouger quelques constantes, mais toutes les mathématiques pareilles.  

\begin{example}
Soit
\begin{equation*}
f(t) =
\lvert t \rvert
\qquad \text{for } \; {-1} < t \leq 1,
\end{equation*}
prolongée périodiquement.  Le graphe est donné par la \figureref{gfs:sawcontfig}.
Calculons la série de Fourier de $f(t)$.

\begin{myfig}
\capstart
\diffyincludegraphics{width=3in}{width=4.5in}{gfs-sawcont}
\caption{Prolongement périodique de la fonction $f(t)$.\label{gfs:sawcontfig}}
\end{myfig}

On veut écrire $f(t) = \frac{a_0}{2} + \sum_{n=1}^\infty a_n \cos (n \pi t) + b_n
\sin (n \pi t)$.  Pour $n \geq 1$ on note que $\lvert t \rvert \cos (n \pi t)$
est paire et aussi
\begin{equation*}
\begin{split}
a_n & = \int_{-1}^1 f(t) \cos (n \pi t) \, dt \\
& = 2 \int_{0}^1 t \cos (n \pi t) \, dt \\
 & = 2 \left[ \frac{t}{n \pi} \sin (n \pi t) \right]_{t=0}^1 -
2 \int_{0}^1 \frac{1}{n \pi} \sin (n \pi t) \, dt \\
& =  0 + \frac{1}{n^2 \pi^2} \Bigl[ \cos (n \pi t) \Bigr]_{t=0}^1
 =  \frac{2 \bigl( {(-1)}^n -1 \bigr) }{n^2 \pi^2}
=
\begin{cases}
0 & \text{if } n \text{ est paire} , \\
\frac{-4 }{n^2 \pi^2} & \text{if } n \text{est impaire}  .
\end{cases}
\end{split}
\end{equation*}
Ensuite, on trouve $a_0$:
\begin{equation*}
a_0 = \int_{-1}^1 \lvert t \rvert \, dt 
=
1 .
\end{equation*}
On devrait être en mesure de trouver cette intégrale en pensant à l'intégrale comme l'aire sous la courbe du graphique sans pour autant faire les calculs. Finalement, on trouve $b_n$.  Ici, on remarque que
$\lvert t \rvert \sin (n \pi t)$ est impaire, et par conséquent: 
\begin{equation*}
b_n = \int_{-1}^1 f(t) \sin (n \pi t) \, dt = 0 .
\end{equation*}
Ainsi, la série est
\begin{equation*}
\frac{1}{2} + 
\sum_{\substack{n=1 \\ n \text{ odd}}}^\infty \frac{-4}{n^2 \pi^2} \cos (n \pi t) .
\end{equation*}

Écrivons explicitement les premiers termes de la série jusqu'à la troisième harmonique.
\begin{equation*}
\frac{1}{2} -
\frac{4}{\pi^2} \cos (\pi t)
-
\frac{4}{9 \pi^2} \cos (3 \pi t)
- \cdots
\end{equation*}
Voir \figureref{gfs:sawcontfsfig}.  On devrait remarquer à quel point le graphe est proche de la vraie fonction. On devrait aussi remarquer qu'il n'y a pas de 
\myquote{phénomène de Gibbs} puisqu'il n'y a pas de discontinuité.

\begin{myfig}
\capstart
%original files gfs-sawcontfs3 gfs-sawcont-fs20
\diffyincludegraphics{width=6.24in}{width=9in}{gfs-sawcont-fs3-fs20}
\caption{La série de Fourier $f(t)$ jusqu'à la troisième harmonique (à gauche)
et jusqu'à la vingtième harmonique (à droite).\label{gfs:sawcontfsfig}}
\end{myfig}
\end{example}

\subsection{Convergence}

On aura besoin de la limite d'un côté de la fonction. On utilisera la notation suivante 
\begin{equation*}
f(c-) = \lim_{t \uparrow c} f(t),
\qquad \text{et} \qquad
f(c+) = \lim_{t \downarrow c} f(t).
\end{equation*}
Si l'on n'est pas familier avec cette notation, 
$\lim_{t \uparrow c} f(t)$ signifie qu'on prend la limite de $f(t)$
lorsque $t$ tend vers $c$ par le bas (i.e.\ $t < c$) et
$\lim_{t \downarrow c} f(t)$ signifie qu'on prend la limite de $f(t)$
lorsque $t$ tend vers $c$ par le haut (i.e.\ $t > c$).
Par exemple, pour la vague carrée 
\begin{equation} \label{gfs:sqwaveeq}
f(t) =
\begin{cases}
0 & \text{si } \; {-\pi} < t \leq 0 , \\
\pi & \text{si } \; \phantom{-}0 < t \leq \pi ,
\end{cases}
\end{equation}
sur $f(0-) = 0$ et $f(0+) = \pi$.

Soit $f(t)$ une fonction définit sur l'intervalle $[a,b]$.  On suppose trouver un nombre de points fini 
$a=t_0$, $t_1$, $t_2$, \ldots, $t_k=b$ dans l'intervalle, tel que  $f(t)$ est continue sur les intervalles
  $(t_0,t_1)$, 
$(t_1,t_2)$, \ldots, 
$(t_{k-1},t_k)$.
On suppose aussi que toutes les limites d'un côté existent, ce qui signifie que tous les 
$f(t_0+)$,
$f(t_1-)$,
$f(t_1+)$,
$f(t_2-)$,
$f(t_2+)$,
\ldots,
$f(t_k-)$
existent et sont finis.
Ensuite, on dit que $f(t)$ est \emph{\myindex{continue par morceaux}}.

De plus, si $f(t)$ est différentiable partout sauf sur un nombre de points finis et que $f'(t)$ est continue par morceaux, alors
$f(t)$ est \emph{\myindex{lisse par morceaux}}.

\begin{example}
La vague carrée \eqref{gfs:sqwaveeq}
est lisse par morceaux sur  $[-\pi,\pi]$ ou tout autre intervalle. Dans un tel cas, on dit simplement que la fonction est lisse par morceaux.  
\end{example}

\begin{example}
La fonction $f(t) = \lvert t \lvert$
est lisse par morceaux.
\end{example}

\begin{example}
La fonction $f(t) = \frac{1}{t}$ n'est pas lisse par morceaux sur 
$[-1,1]$ (ou sur n'importe quel intervalle contenant zéro). En effet, elle n'est même pas continue par morceaux. 
\end{example}

\begin{example}
La fonction $f(t) = \sqrt[3]{t}$ n'est pas lisse par morceaux sur 
$[-1,1]$ (ou sur n'importe quel intervalle contenant zéro).  Elle est continue, mais la dérivée de $f(t)$ est non bornée près de zéro et ainsi pas continue par morceaux. 
\end{example}

\begin{theorem}
Supposons que  $f(t)$ est une fonction $2L$-périodique lisse par morceaux.
Soit
\begin{equation*}
\frac{a_0}{2} + \sum_{n=1}^\infty a_n \cos \left( \frac{n \pi}{L} t
\right)
+ b_n \sin \left( \frac{n \pi}{L} t \right)
\end{equation*}
une série de Fourier pour $f(t)$.  Alors la série converge pour tout $t$.  Si $f(t)$ est continue à $t$,  alors
\begin{equation*}
f(t) = \frac{a_0}{2} + \sum_{n=1}^\infty
a_n \cos \left( \frac{n \pi}{L} t \right)
+ b_n \sin \left( \frac{n \pi}{L} t \right) .
\end{equation*}
Autrement,
\begin{equation*}
\frac{f(t-)+f(t+)}{2} =
\frac{a_0}{2} + \sum_{n=1}^\infty a_n \cos \left( \frac{n \pi}{L}  t
\right)
+ b_n \sin \left( \frac{n \pi}{L} t \right) .
\end{equation*}
\end{theorem}

S'il arrive qu'on aille
$f(t) = \frac{f(t-)+f(t+)}{2}$ à toutes les discontinuités, la série de Fourier converge à  $f(t)$ partout. On peut toujours simplement redéfinir $f(t)$ en changeant la valeur à chaque discontinuité de manière appropriée. Alors on peut écrire un  signe d'égalité entre $f(t)$ et la série sans avoir peur. On a mentionné  brièvement ce fait à la fin de la section précédente. 

Le théorème ne dit pas la vitesse à laquelle la série converge. En retournant en arrière à la dernière section à la discussion sur le phénomène Gibbs: plus on se rapproche de la discontinuité, plus on a besoin de termes pour obtenir une approximation acceptable de la fonction.  

\subsection{Dérivation et intégration de séries de Fourier}

Non seulement la série de Fourier converge bien, mais elle est facile à différencier
et à intégrer. Nous pouvons la faire simplement en différenciant ou en intégrant terme par terme.

\begin{theorem}
Supposons
\begin{equation*}
f(t) = \frac{a_0}{2} + \sum_{n=1}^\infty a_n \cos \left( \frac{n \pi}{L} t
\right)
+ b_n \sin \left( \frac{n \pi}{L} t \right)
\end{equation*}
est une fonction continue lisse par morceaux et la dérivée $f'(t)$ est  lisse
par morceaux. Alors la dérivée peut être
obtenue en différenciant terme par terme.

\begin{equation*}
f'(t) = \sum_{n=1}^\infty \frac{-a_n n \pi}{L} 
\sin \left( \frac{n \pi}{L} t \right)
+ \frac{b_n n \pi}{L} \cos \left( \frac{n \pi}{L} t \right) .
\end{equation*}
\end{theorem}

Il est important que la fonction soit continue. Il peut avoir des coins, mais pas de
sauts. Sinon, la dérivée de la série ne parviendra pas à converger. Comme 
exercice, prenez la série obtenue pour la vague carrée et essayez de la
différencier. De même, on peut également intégrer une série de Fourier.

\begin{theorem}
Supposons
\begin{equation*}
f(t) = \frac{a_0}{2} + \sum_{n=1}^\infty
a_n \cos \left( \frac{n \pi}{L} t \right)
+ b_n \sin \left( \frac{n \pi}{L} t \right)
\end{equation*}
est une fonction lisse par morceaux. Alors la primitive est
obtenue en intégrant terme par terme et ainsi
\begin{equation*}
F(t) = \frac{a_0 t}{2} + C + \sum_{n=1}^\infty
\frac{a_n L}{n \pi} \sin \left( \frac{n \pi}{L} t \right)
+ \frac{-b_n L}{n \pi}  \cos \left( \frac{n \pi}{L} t \right) ,
\end{equation*}
où  $F'(t) = f(t)$ et $C$ est une constante arbitraire. 
\end{theorem}

On note que la série pour $F(t)$ n'est plus une série de Fourier car elle contient le terme
 $\frac{a_0 t}{2}$.  La primitive d'une fonction périodique n'a plus besoin d'être périodique et l'on ne devrait pas
s'attendre à une série de Fourier.

\subsection{Taux de convergence et différentiabilité}

Considérons un exemple de fonction périodique qui se dérive partout.

\begin{example}
Prenons la fonction
\begin{equation*}
f(t) =
\begin{cases}
(t+1)\,t & \text{si } \; {-1} < t \leq 0 , \\
(1-t)\,t & \text{si } \; \phantom{-}0 < t \leq 1 ,
\end{cases}
\end{equation*}
et prolongeons-la à une fonction 
2-périodique.  Voir~\figureref{gfs:smoothexfig}.

\begin{myfig}
\capstart
\diffyincludegraphics{width=3in}{width=4.5in}{gfs-smoothex}
\caption{Fonction 2-périodique lisse.\label{gfs:smoothexfig}}
\end{myfig}

Cette fonction est dérivable partout, mais elle
n'a pas de dérivée seconde à chaque $t$ entier.

\begin{exercise}
Calculez  $f''(0+)$ et $f''(0-)$.
\end{exercise}

Voici les coefficients de la série de Fourier.  Leur calcul 
implique plusieurs intégrations par parties et est laissé à l'étudiant.

\begin{align*}
a_0 & = 
\int_{-1}^1
f(t) \, dt = 
\int_{-1}^0
(t+1)\,t \, dt +
\int_0^1
(1-t)\,t \, dt = 0 , \\
a_n & = 
\int_{-1}^1
f(t) \, \cos (n\pi t) \, dt = 
\int_{-1}^0
(t+1)\,t
\, \cos (n \pi t) \, dt +
\int_0^1
(1-t)\,t
\, \cos (n \pi t) \, dt = 0, \\
b_n & = 
\int_{-1}^1
f(t) \, \sin (n\pi t) \, dt = 
\int_{-1}^0
(t+1)\,t
\, \sin (n \pi t) \, dt +
\int_0^1
(1-t)\,t
\, \sin (n \pi t) \, dt \\
& =
\frac{4 ( 1-{(-1)}^n)}{\pi^3 n^3} 
=
\begin{cases}
\frac{8}{\pi^3 n^3} & \text{si } n \text{ est impaire} , \\
0 & \text{si } n \text{ est paire} .
\end{cases}
\end{align*}
Alors la série est 
\begin{equation*}
\sum_{\substack{n=1 \\ n \text{ odd}}}^\infty \frac{8}{\pi^3 n^3} \sin (n \pi t) .
\end{equation*}

Cette série converge très vite.
Si vous calculez jusqu'à la troisième harmonique, c'est la fonction
\begin{equation*}
\frac{8}{\pi^3} \sin (\pi t) + 
\frac{8}{27 \pi^3} \sin (3 \pi t) ,
\end{equation*}

Son graphe est presque impossible à distinguer de celui de $f(t)$ dans la 
\figureref{gfs:smoothexfig}.
En effet, le coefficient
$\frac{8}{27 \pi^3}$ est déjà juste 0.0096 (approximativement).
La raison de ce comportement est le terme $n^3$ au dénominateur.
le coefficient $b_n$, dans ce cas, va à zéro aussi vite que
$\nicefrac{1}{n^3}$ va à zéro.
\end{example}

Pour les fonctions construites par morceaux à partir de polynômes comme ci-dessus,
il est généralement vrai que si l'on a une dérivée, les coefficients de la série de Fourier
tendront vers zéro approximativement comme $\nicefrac{1}{n^3}$.  Si l'on n'a
 qu'une fonction continue, alors les coefficients de Fourier tendront vers zéro comme $\nicefrac{1}{n^2}$.  Si l'on a des discontinuités, alors
les coefficients de Fourier tendront vers zéro approximativement comme $\nicefrac{1}{n}$.
Pour des fonctions plus générales, l'histoire est un peu plus compliquée mais c'est la
même idée; plus on a de dérivées, plus les coefficients convergent rapidement vers zéro. Un raisonnement similaire fonctionne en sens inverse. Si les coefficients tendent vers 
zéro comme $\nicefrac{1}{n^2}$, on obtient toujours une fonction continue. S'ils tendent vers zéro comme $\nicefrac{1}{n^3}$, 
on obtient une fonction différentiable partout.  Les coefficients de Fourier nous en disent long sur la différentiabilité de la fonction. 
%Therefore, we can tell a lot about the smoothness of a function by looking
%at its Fourier coefficients.

Pour justifier ce comportement, prenons, par exemple, la fonction définie par
la série Fourier
\begin{equation*}
f(t) = \sum_{n=1}^\infty \frac{1}{n^3} \sin (n t) .
\end{equation*}

Lorsqu'on dérive terme par terme, on remarque
\begin{equation*}
f'(t) = \sum_{n=1}^\infty \frac{1}{n^2} \cos (n t) .
\end{equation*}
Par conséquent, les coefficients diminuent maintenant comme $\nicefrac{1}{n^2}$, ce qui 
signifie que nous avons une fonction continue.
La dérivée de $f'(t)$ est définie à la plupart des points, mais il y a des points où $f'(t)$ n'est pas différentiable.
Il y a des coins, mais pas de sauts.
Si l'on dérive à nouveau (là où on peut), on constate que la fonction
$f''(t)$ n'est pas continue (elle a des sauts).
\begin{equation*}
f''(t) = \sum_{n=1}^\infty \frac{-1}{n} \sin (n t) .
\end{equation*}
Cette fonction est similaire à la dent de scie. Si l'on essayait de dériver 
la série à nouveau, on obtiendrait
\begin{equation*}
\sum_{n=1}^\infty -\cos (n t) ,
\end{equation*}
qui ne converge pas!

\begin{exercise}
Utilisez un ordinateur pour tracer la série qu'on a obtenue pour $f(t)$, $f'(t)$ et
$f''(t)$.  Autrement dit, tracer les 5 premières harmoniques des fonctions. À quels
 points  $f''(t)$ a-t-elle des discontinuités?
\end{exercise}

\subsection{Exercices}

\begin{exercise}
Soit
\begin{equation*}
f(t) =
\begin{cases}
0 & \text{si } \; {-1} < t \leq 0 , \\
t & \text{si } \; \phantom{-}0 < t \leq  1 
\end{cases}
\end{equation*}
prolongée périodiquement.
\begin{tasks}
\task Calculez la série de Fourier pour $f(t)$.
\task Écrivez la série explicitement jusqu'à la troisième harmonique.
\end{tasks}
\end{exercise}

\begin{exercise}
Soit
\begin{equation*}
f(t) =
\begin{cases}
-t & \text{si } \; {-1} < t \leq 0 , \\
t^2 & \text{si } \; \phantom{-}0 < t \leq  1 
\end{cases}
\end{equation*}
prolongée périodiquement.

\begin{tasks}
\task Calculez la série de Fourier pour $f(t)$.
\task Écrivez la série explicitement jusqu'à la troisième harmonique.
\end{tasks}
\end{exercise}

\begin{exercise}
Soit
\begin{equation*}
f(t) =
\begin{cases}
\frac{-t}{10} & \text{si } \; {-10} < t \leq 0 , \\
\frac{t}{10} & \text{si } \; \phantom{-1}0 < t \leq  10 ,
\end{cases}
\end{equation*}
prolongée périodiquement (la période est 20).
\begin{tasks}
\task Calculez la série de Fourier pour $f(t)$.
\task Écrivez la série explicitement jusqu'à la troisième harmonique.
\end{tasks}
\end{exercise}

\begin{exercise}
Soit  $f(t) = \sum_{n=1}^\infty \frac{1}{n^3} \cos (n t)$.  Est-ce que $f(t)$
est continue et différentiable partout? Trouvez la dérivée (si elle existe partout) ou justifiez pourquoi $f(t)$ n'est pas différentiable partout.
\end{exercise}

\begin{exercise}
Soit $f(t) = \sum_{n=1}^\infty \frac{{(-1)}^n}{n} \sin (n t)$.  Est-ce que  $f(t)$
est différentiable partout?  Trouvez la dérivée (si elle existe partout ou justifiez pourquoi $f(t)$ n'est pas différentiable partout).
\end{exercise}

\begin{exercise}
Soit
\begin{equation*}
f(t) =
\begin{cases}
0 & \text{si } \; {-2} < t \leq 0, \\
t & \text{si } \; \phantom{-}0 < t \leq 1, \\
-t+2 & \text{si } \; \phantom{-}1 < t \leq 2,
\end{cases}
\end{equation*}
prolongée périodiquement.
\begin{tasks}
\task Calculez la série de Fourier pour $f(t)$.
\task Écrivez la série explicitement jusqu'à la troisième harmonique.
\end{tasks}
\end{exercise}

\begin{exercise}
Soit
\begin{equation*}
f(t) = e^t \qquad \text{pour } \; {-1} < t \leq 1
\end{equation*}
prolongée périodiquement.
\begin{tasks}
\task Calculez la série de Fourier pour $f(t)$.
\task Écrivez la série explicitement jusqu'à la troisième harmonique.
\task Vers quoi la série converge-t-elle à $t=1$?
\end{tasks}
\end{exercise}

\begin{exercise}
Soit
\begin{equation*}
f(t) = t^2 \qquad \text{pour } \; {-1} < t \leq 1
\end{equation*}
prolongée périodiquement.
\begin{tasks}
\task Calculez la série de Fourier pour $f(t)$.
\task En remplaçant $t=0$,
évaluez $\displaystyle \sum_{n=1}^\infty \frac{{(-1)}^n}{n^2} = 1 - \frac{1}{4} +
\frac{1}{9} - \cdots$.
\task Maintenant évaluez $\displaystyle \sum_{n=1}^\infty \frac{1}{n^2} = 1 + \frac{1}{4} +
\frac{1}{9} + \cdots$.
\end{tasks}
\end{exercise}

\begin{exercise}
Soit
\begin{equation*}
f(t) =
\begin{cases}
0 & \text{si } \; {-3} < t \leq 0, \\
t & \text{si } \; \phantom{-}0 < t \leq 3,
\end{cases}
\end{equation*}
prolongée périodiquement.  Supposez que $F(t)$ est la fonction donnée
par la série de Fourier de $f$.  Sans calculer la série de Fourier
évaluez
\begin{tasks}(3)
\task $F(2)$
\task $F(-2)$
\task $F(4)$
\task $F(-4)$
\task $F(3)$
\task $F(-9)$
\end{tasks}
\end{exercise}

\setcounter{exercise}{100}

\begin{exercise}
Soit
\begin{equation*}
f(t) = t^2 \qquad \text{pour } \; {-2} < t \leq 2
\end{equation*}
prolongée périodiquement.
\begin{tasks}
\task Calculez la série de Fourier pour $f(t)$.
\task Écrivez la série explicitement jusqu'à la troisième harmonique.
\end{tasks}
\end{exercise}
\exsol{%
a) $\frac{8}{6} +
\sum\limits_{n=1}^\infty
\frac{16{(-1)}^n}{\pi^2 n^2}
\cos\bigl(\frac{n\pi}{2} t\bigr)$
\quad
b) $\frac{8}{6}
-
\frac{16}{\pi^2 }
\cos\bigl(\frac{\pi}{2} t\bigr)
+
\frac{4}{\pi^2}
\cos\bigl(\pi t\bigr)
-
\frac{16}{9\pi^2}
\cos\bigl(\frac{3\pi}{2} t\bigr) + \cdots$
}

\begin{exercise}
Soit
\begin{equation*}
f(t) = t \qquad \text{pour } \; {-\lambda} < t \leq \lambda \; \text{ (pour un certain} \lambda > 0 \text{)}
\end{equation*}
prolongée périodiquement.
\begin{tasks}
\task Calculez la série de Fourier pour $f(t)$.
\task Écrivez la série explicitement jusqu'à la troisième harmonique.
\end{tasks}
\end{exercise}
\exsol{%
a)
$\sum\limits_{n=1}^\infty
\frac{{(-1)}^{n+1}2\lambda}{n \pi}
\sin\bigl(\frac{n\pi}{\lambda} t\bigr)$
\quad
b)
$\frac{2\lambda}{\pi}
\sin\bigl(\frac{\pi}{\lambda} t\bigr)
-
\frac{\lambda}{\pi}
\sin\bigl(\frac{2\pi}{\lambda} t\bigr)
+
\frac{2\lambda}{3\pi}
\sin\bigl(\frac{3\pi}{\lambda} t\bigr) - \cdots$
}

\begin{exercise}
Soit
\begin{equation*}
f(t) = \frac{1}{2} + \sum_{n=1}^\infty
\frac{1}{n(n^2+1)}
\sin(n\pi t) .
\end{equation*}
Calculez $f'(t)$.
\end{exercise}
\exsol{%
$f'(t) = \sum\limits_{n=1}^\infty
\frac{\pi}{n^2+1}
\cos(n\pi t)$
}

\begin{exercise}
Soit
\begin{equation*}
f(t) = \frac{1}{2} + \sum_{n=1}^\infty
\frac{1}{n^3}
\cos(n t) .
\end{equation*}
\begin{tasks}
\task Trouvez la primitive.
\task La primitive est-elle périodique?
\end{tasks}
\end{exercise}
\exsol{%
a)
$F(t) = \frac{t}{2} + C + \sum\limits_{n=1}^\infty
\frac{1}{n^4}
\sin(nt)$
\qquad
b) non
}

\begin{exercise}
Soit
\begin{equation*}
f(t) = \nicefrac{t}{2} \qquad \text{pour } \; {-\pi} < t < \pi
\end{equation*}
prolongée périodiquement.
\begin{tasks}
\task Calculez la série de Fourier pour $f(t)$.
\task Remplacez $t=\nicefrac{\pi}{2}$ pour trouver une représentation en série
pour $\nicefrac{\pi}{4}$.
\task En utilisant les 4 premiers termes du résultat de la partie b) approximez
$\nicefrac{\pi}{4}$.
\end{tasks}
\end{exercise}
\exsol{%
a)
$\sum\limits_{n=1}^\infty
\frac{{(-1)}^{n+1}}{n} \sin(nt)$
\qquad
b) $f$ est continue à $t=\nicefrac{\pi}{2}$ alors le
la série de Fourier converge vers $f(\nicefrac{\pi}{2}) = \nicefrac{\pi}{4}$.
On obtient
$\nicefrac{\pi}{4} = \sum\limits_{n=1}^\infty
\frac{{(-1)}^{n+1}}{2n-1} = 1 - \nicefrac{1}{3} + \nicefrac{1}{5}-
\nicefrac{1}{7} + \cdots$.
\qquad
c) En utilisant les 4 premiers termes, on obtient $\nicefrac{76}{105}\approx 0.72$ (une assez mauvaise
approximation, vous devrez prendre environ 50 termes pour commencer à arriver à
 $0.01$ de $\nicefrac{\pi}{4}$).}

\begin{exercise}
Soit
\begin{equation*}
f(t) = 
\begin{cases}
0 & \text{si } \; {-2} < t \leq 0, \\
2 & \text{si } \; \phantom{-}0 < t \leq 2,
\end{cases}
\end{equation*}
prolongée périodiquement.  Supposez que  $F(t)$ soit la fonction donnée
par la série de Fourier de $f$.  Sans calculer la série de Fourier
évaluez
\begin{tasks}(3)
\task $F(0)$
\task $F(-1)$
\task $F(1)$
\task $F(-2)$
\task $F(4)$
\task $F(-8)$
\end{tasks}
\end{exercise}
\exsol{%
a) $F(0) = 1$, 
b) $F(-1) = 0$, 
c) $F(1) = 2$, 
d) $F(-2) = 1$, 
e) $F(4) = 1$, 
f) $F(-9) = 0$ 
}

%%%%%%%%%%%%%%%%%%%%%%%%%%%%%%%%%%%%%%%%%%%%%%%%%%%%%%%%%%%%%%%%%%%%%%%%%%%%%%

\sectionnewpage
\section{Séries de sinus et de cosinus}
\label{sec:scs}

%\sectionnotes{2 lectures\EPref{, \S9.3 in \cite{EP}}\BDref{,
%\S10.4 in \cite{BD}}}

\subsection{Fonctions périodiques paires et impaires}

Vous avez peut-être remarqué qu'une fonction impaire n'a pas de termes en cosinus dans la série de Fourier et qu'une fonction paire n'a pas de termes en sinus dans la série de Fourier.
Cette observation n'est pas une coïncidence. Regardons les fonctions périodiques paires et impaires plus en détail.

Rappelons-nous qu'une fonction $f(t)$ est \emph{impaire}\index{odd function} si $f(-t) =
-f(t)$.  Une fonction $f(t)$ est \emph{paire}\index{even function} si
$f(-t) = f(t)$.  Par exemple, $\cos (n t)$ est paire et  $\sin (n t)$ est impaire.
De même, la fonction $t^k$est paire si $k$ est paire et impaire si $k$ est impaire.

\begin{exercise}
Prenez deux fonctions $f(t)$ et $g(t)$ et définissez leur produit $h(t) =
f(t)g(t)$.
\begin{tasks}
\task Supposez que $f(t)$ et $g(t)$ sont impaires.  Est-ce que $h(t)$ est impaire ou paire?
\task Supposez que l'une est paire et l'autre est impaire. Est-ce que $h(t)$ est impaire ou paire?
\task Supposez que les deux sont paires. Est-ce que $h(t)$ est impaire ou paire?
\end{tasks}
\end{exercise}

Si $f(t)$ et $g(t)$ sont toutes les deux impaires, alors $f(t)+g(t)$ est impaire.  De même, pour
des fonctions paires. D'autre part,
si  $f(t)$ est impaire et $g(t)$est paire,  alors on ne peut rien dire sur la somme
$f(t) + g(t)$.  En fait, une série de Fourier est une somme d'une fonction impaire (les termes sinus) et d'une fonction paire (les termes cosinus).

Dans cette section, on considère les fonctions périodiques impaires et paires. On a défini précédemment l'extension $2L$-periodique de la fonction définie sur l'intervalle $[-L,L]$.  Parfois, on s'intéresse seulement aux fonctions dans l'intervalle $[0,L]$ et ce serait pratique pour avoir une fonction impaire (resp. \ paire). Si la fonction est impaire (resp. \ paire), tous les termes cosinus (resp. \ sine) disparaissent.
Ce que nous allons faire est prendre l'extension impaire (resp. \ paire) de la fonction à $[-L,L]$ 
puis l'étendre périodiquement à une fonction $2L$-périodique.  

On prend une fonction $f(t)$ définie sur $[0,L]$.  Sur $(-L,L]$ la fonction est définie.
\begin{align*}
F_{\text{impaire}}(t) & \overset{\text{def}}{=}
\begin{cases}
f(t) & \text{si } \; \phantom{-}0 \leq t \leq L , \\
-f(-t) & \text{si } \; {-L} < t < 0 ,
\end{cases}
\\
F_{\text{paire}}(t) & \overset{\text{def}}{=}
\begin{cases}
f(t) & \text{si } \; \phantom{-}0 \leq t \leq L , \\
f(-t) & \text{si } \; {-L} < t < 0 .
\end{cases}
\end{align*}
On étend $F_{\text{impaire}}(t)$ et $F_{\text{paire}}(t)$ à une $2L$-periodique.
Alors
$F_{\text{impaire}}(t)$ est appelée l' \emph{\myindex{extension périodique impaire}} de $f(t)$, et
$F_{\text{paire}}(t)$ est appelée l'
\emph{\myindex{extension périodique paire}} de $f(t)$.
Pour l'extension impaire, on suppose généralement que $f(0) = f(L) = 0$.

\begin{exercise}
Vérifiez que $F_{\text{impaire}}(t)$ est impaire et que  $F_{\text{paire}}(t)$ est paire.
Pour $F_{\text{impaire}}$,
assumez $f(0) = f(L) = 0$.
\end{exercise}

\begin{example}
Prenez la fonction $f(t) = t\,(1-t)$ definie sur $[0,1]$. 
La \figurevref{scs:oddevenextfig}
montre les graphiques des extensions périodiques impaires et paires de $f(t)$.

\begin{myfig}
\capstart
%original files scs-oddext scs-evenext
\diffyincludegraphics{width=6.24in}{width=9in}{scs-ext-odd-even}
\caption{Extension 2-périodique impaire et paire de $f(t) =
t\,(1-t)$, $0 \leq t \leq 1$.\label{scs:oddevenextfig}}
\end{myfig}
\end{example}

\subsection{Séries sinus et cosinus}

Soit $f(t)$ une fonction $2L$-periodique impaire.  On écrit
la série de Fourier pour $ f (t) $. Tout d'abord, on calcule les coefficients $a_n$ (incluant
$n=0$) et l'on obtient
\begin{equation*}
a_n = \frac{1}{L} \int_{-L}^L f(t) \cos \left( \frac{n \pi}{L} t \right)
\, dt = 0 .
\end{equation*}
Autrement dit, il n'y a pas de terme cosinus dans la série de Fourier d'une fonction impaire.
L'intégrale est nulle, car $f(t) \cos \left( {n \pi}{L} t \right)$
est une fonction impaire (le produit d'une fonction impaire et d'une
fonction paire est impaire) et l'intégrale d'une fonction impaire sur un intervalle symétrique est toujours égal à zéro. L'intégrale d'une fonction paire sur un intervalle symétrique
$[-L,L]$  est le double de l'intégrale de la fonction sur l'intervalle $[0,L]$.
La fonction $f(t) \sin \left( \frac{n \pi}{L} t \right)$ est le produit de deux fonctions impaires
et est donc paire.
\begin{equation*}
b_n = 
\frac{1}{L} \int_{-L}^L f(t) \sin \left( \frac{n \pi}{L} t \right) \, dt =
\frac{2}{L} \int_{0}^L f(t) \sin \left( \frac{n \pi}{L} t \right) \, dt .
\end{equation*}
On écrit maintenant la série de Fourier de $f(t)$ comme
\begin{equation*}
\sum_{n=1}^\infty b_n \sin \left( \frac{n \pi}{L} t \right) .
\end{equation*}

De même, si $f(t)$ est une fonction  $2L$-périodique paire.  Pour le même 
raison que ci-dessus, on constate que $b_n = 0$ et
\begin{equation*}
a_n = 
\frac{2}{L} \int_{0}^L f(t) \cos \left( \frac{n \pi}{L} t \right) \, dt .
\end{equation*}
La formule fonctionne toujours pour $n=0$, auquel cas il devient
\begin{equation*}
a_0 = 
\frac{2}{L} \int_{0}^L f(t) \, dt .
\end{equation*}
La série Fourier est alors
\begin{equation*}
\frac{a_0}{2}
+
\sum_{n=1}^\infty a_n \cos \left( \frac{n \pi}{L} t \right) .
\end{equation*}

Une conséquence intéressante est que les coefficients de la série de Fourier d'une fonction impaire (ou paire) peuvent être calculés en intégrant simplement sur la moitié de l'intervalle $[0,L]$.  Par conséquent, on peut calculer la série de Fourier de
l'extension impaire (ou paire) d'une fonction en calculant certaines intégrales sur l'intervalle
où la fonction d'origine est définie.

\begin{theorem}
Soit $f(t)$ une fonction lisse par morceaux définie sur $[0,L]$.
Puis l'extension périodique impaire
de $f(t)$ a la série Fourier
\begin{equation*}
\mybxbg{~~
F_{\text{odd}}(t) = \sum_{n=1}^\infty b_n \sin \left( \frac{n \pi}{L} t
\right) ,
~~}
\end{equation*}
où
\begin{equation*}
\mybxbg{~~
b_n = 
\frac{2}{L} \int_{0}^L f(t)\, \sin \left( \frac{n \pi}{L} t \right) \, dt .
~~}
\end{equation*}
L'extension périodique paire de $f (t)$ a la série de Fourier
\begin{equation*}
\mybxbg{~~
F_{\text{even}}(t) = \frac{a_0}{2} + \sum_{n=1}^\infty a_n \cos \left(
\frac{n \pi}{L} t \right) ,
~~}
\end{equation*}
où
\begin{equation*}
\mybxbg{~~
a_n = 
\frac{2}{L} \int_{0}^L f(t)\, \cos \left( \frac{n \pi}{L} t \right) \, dt .
~~}
\end{equation*}
\end{theorem}

On appelle la série $\sum_{n=1}^\infty b_n \sin \left( \frac{n \pi}{L} t\right)$ 
la \emph{\myindex{série sinus}} de  $f(t)$ et l'on appelle la série
$\frac{a_0}{2} + \sum_{n=1}^\infty a_n \cos \left( \frac{n \pi}{L} t
\right)$
la \emph{\myindex{série cosinus}} de $f(t)$.  
On ne se soucie pas souvent de ce qui se passe en dehors de $ [0, L] $. Dans ce cas,
on choisit la série qui correspond le mieux au problème.

Il n'est pas nécessaire de commencer par la série de Fourier complète pour obtenir
les séries de sinus et de cosinus. La série de sinus est en réalité le développement des fonctions propres de $ f (t) $ en utilisant les
fonctions propres du problème des valeurs propres $x''+\lambda x = 0$, $x(0) = 0$,
$x(L) = L$.  La série de cosinus est le développement des fonctions propres de $ f (t) $
en utilisant les fonctions propres du problème des valeurs propres $x''+\lambda x = 0$, $x'(0) = 0$,
$x'(L) = L$.  On aurait donc pu obtenir les mêmes formules
en définissant le produit intérieur
\begin{equation*}
\langle f(t), g(t) \rangle = \int_0^L f(t) g(t) \, dt ,
\end{equation*}
et en suivant la procédure de \sectionref{ts:section}.  Ce point de vue est
utile, car on utilise couramment une série spécifique qui est née de la question sous-jacente qui a
conduit à un certain problème de valeur propre. Si la valeur propre du
problème n'est pas l'une des trois que nous avons abordées jusqu'à présent, on peut toujours faire un
développement des fonctions propres en généralisant les résultats de ce chapitre. (Voir les références pour en savoir plus long.)
%On va
%faire face à une telle généralisation dans le \chapterref{SL:chapter}.

%f(t) = \frac{a_0}{2} + \sum_{n=1}^\infty a_n \cos \left( \frac{n \pi}{L} 
%t \right)
%+ b_n \sin \left( \frac{n \pi}{L} t \right) ,

\begin{example}
Trouvez la série de Fourier de l'extension périodique paire de
la fonction $f(t) = t^2$ pour $0 \leq t \leq \pi$.

On veut écrire
\begin{equation*}
f(t) = \frac{a_0}{2} + \sum_{n=1}^\infty a_n \cos (n t) ,
\end{equation*}
où
\begin{equation*}
a_0 = \frac{2}{\pi}
\int_0^\pi t^2 \, dt = \frac{2 \pi^2}{3} ,
\end{equation*}
et
\begin{equation*}
\begin{split}
a_n & = \frac{2}{\pi}
\int_0^\pi t^2 \cos (n t) \, dt
= \frac{2}{\pi} \left[ t^2 \frac{1}{n} \sin (nt) \right]_0^\pi -
\frac{4}{n\pi}
\int_0^\pi t \sin (n t) \, dt \\
& = 
\frac{4}{n^2\pi}
\Bigl[ t \cos (n t) \Bigr]_0^\pi
+
\frac{4}{n^2\pi}
\int_0^\pi \cos (n t) \, dt
= 
\frac{4{(-1)}^n}{n^2} .
\end{split}
\end{equation*}
Notez qu'on a \myquote{detecté} la continuité de l'extension depuis le
les coefficients décroissants comme $ \ frac {1} {n ^ 2} $. Autrement dit, l'extension périodique paire
de $ t ^ 2 $ n'a pas de discontinuités de saut. Elle a des coins, car
la dérivée, qui est une fonction impaire et une série sinus, a des sauts; il existe
une série de Fourier dont les coefficients sont seulement décroissants comme $ \ frac {1} {n} $. 

Explicitement, les premiers termes de la série sont
\begin{equation*}
\frac{\pi^2}{3} - 4 \cos (t) + \cos (2t) - \frac{4}{9} \cos (3t) + \cdots
\end{equation*}
\end{example}

\begin{exercise}
\leavevmode
\begin{tasks}
\task Calculez la dérivée de l'extension périodique paire de $ f (t) $ ci-dessus et vérifiez s'il y a des sauts de discontinuité. Utilisez la définition réelle de $ f (t) $, pas sa série cosinus!
\task Pourquoi est-ce que la dérivée de l'extension périodique paire de $ f (t) $ est l'extension périodique impaire de $ f '(t) $?
\end{tasks}
\end{exercise}

\subsection{Application}

La série de Fourier est liée aux problèmes de valeurs limites
qu'on a étudiés plus tôt. Voyons cette connexion dans une application.

Considérons le problème de valeur limite pour $0 < t < L$,
\begin{equation*}
x''(t) + \lambda\, x(t) = f(t) ,
\end{equation*}
pour les \emph{\myindex{conditions aux frontières de Dirichlet}}
$x(0) = 0$, $x(L) = 0$.
L'alternative de Fredholm(\thmvref{thm:fredholmsimple})
dit que tant que $ \ lambda $ n'est pas une valeur propre du problème homogène sous-jacent, il existe une solution unique.
Les fonctions propres de ce problème de valeurs propres sont les fonctions
$\sin \left( \frac{n \pi}{L} t \right)$.
Donc, pour trouver la solution, on trouve d'abord la série sinus de Fourier pour $ f (t) $.
Ont écrit $ x $ également sous forme de série sinus, mais avec des coefficients inconnus.
On substitue la série pour $ x $ dans l'équation et l'on résout avec les coefficients inconnus.
Si l'on a
la \emph{\myindex{condition à la frontière Neumann}}
$x'(0) = 0$, $x'(L) = 0$, on suit la même procédure en utilisant la série cosinus.

On voit comment cette méthode fonctionne avec des exemples.

\begin{example}
Prenons le problème à valeur limite pour $0 < t < 1$,
\begin{equation*}
x''(t) + 2 x(t) = f(t) ,
\end{equation*}
où $f(t) = t$ on $0 < t < 1$, et satisfaisant à la condition aux frontières de Dirichlet  $x(0) = 0$, $x(1)=0$.
On écrit $f(t)$ comme une série sinus
\begin{equation*}
f(t) = \sum_{n=1}^\infty c_n \sin (n \pi t) .
\end{equation*}
On calcule
\begin{equation*}
c_n = 2 \int_0^1 t \sin (n \pi t) \,dt = \frac{2 \, {(-1)}^{n+1}}{n \pi} .
\end{equation*}
on écrit  $x(t)$ comme
\begin{equation*}
x(t) = \sum_{n=1}^\infty b_n \sin (n \pi t) .
\end{equation*}
On insère et l'on obtient
\begin{equation*}
\begin{split}
x''(t) + 2 x(t) & =
\underbrace{
\sum_{n=1}^\infty - b_n n^2 \pi^2 \sin (n \pi t)
}_{x''}
\,
+
\,
2
\underbrace{
\sum_{n=1}^\infty b_n \sin (n \pi t)
}_{x}
\\
& =
\sum_{n=1}^\infty b_n (2 - n^2 \pi^2 ) \sin (n \pi t)
\\
& = f(t)
=
\sum_{n=1}^\infty  \frac{2\, {(-1)}^{n+1}}{n \pi} \sin (n \pi t) .
\end{split}
\end{equation*}
Donc,
\begin{equation*}
b_n (2 - n^2 \pi^2)
=
\frac{2\,{(-1)}^{n+1}}{n \pi}
\end{equation*}
ou
\begin{equation*}
b_n
=
\frac{2\,{(-1)}^{n+1}}{n \pi (2 - n^2 \pi^2)} .
\end{equation*}
Puisque $ 2 $ n'est pas une valeur propre du problème, on a que $ 2-n ^ 2 \ pi ^ 2 $ n'est pas nul pour tout $ n $, et qu'on peut résoudre pour $ b_n $. On a ainsi obtenu une série de Fourier pour la solution
\begin{equation*}
x(t) = 
\sum_{n=1}^\infty
\frac{2\,{(-1)}^{n+1}}{n \pi \,(2 - n^2 \pi^2)}
\sin (n \pi t) .
\end{equation*}
Regardons \figurevref{bnd-dirich-graph:fig} pour le graphique de la solution.
On note que puisque les fonctions propres satisfont aux conditions aux limites,
et que $ x $ s'écrit en fonction des conditions aux limites, alors $ x $
satisfait aux conditions aux limites
\begin{myfig}
\capstart
\diffyincludegraphics{width=3in}{width=4.5in}{bnd-dirich-graph}
\caption{Graphe de la solution de $x''+2x=t$, $x(0)=0$, $x(1)=0$.%
\label{bnd-dirich-graph:fig}}
\end{myfig}
\end{example}

\begin{example}
De même, on traite les conditions de Neumann.
On prend le problème de la valeur limite pour $0 < t < 1$,
\begin{equation*}
x''(t) + 2 x(t) = f(t) ,
\end{equation*}
où $f(t) = t$ sur $0 < t < 1$, mais maintenant satisfaisant
aux conditions aux limites de Neumann
$x'(0) = 0$, $x'(1)=0$.
On écrit $f(t)$ comme une série cosinue
\begin{equation*}
f(t) = \frac{c_0}{2} + \sum_{n=1}^\infty c_n \cos (n \pi t) ,
\end{equation*}
où
\begin{equation*}
c_0 = 2 \int_0^1 t \,dt = 1 ,
\end{equation*}
et
\begin{equation*}
c_n = 2 \int_0^1 t \cos (n \pi t) \,dt =
\frac{2\bigl({(-1)}^n-1\bigr)}{\pi^2 n^2} = 
\begin{cases}
\frac{-4}{\pi^2 n^2} & \text{if } n \text{ odd} , \\
0 & \text{if } n \text{ even}.
\end{cases}
\end{equation*}
on écrit $x(t)$ comme une série cosinus
\begin{equation*}
x(t) = \frac{a_0}{2} + \sum_{n=1}^\infty a_n \cos (n \pi t) .
\end{equation*}
On insère et l'on obtient
\begin{equation*}
\begin{split}
x''(t) + 2 x(t) & =
\sum_{n=1}^\infty \Bigl[ - a_n n^2 \pi^2 \cos (n \pi t) \Bigr]
+
a_0 +
2
\sum_{n=1}^\infty \Bigl[ a_n \cos (n \pi t) \Bigr]
\\
& =
a_0 +
\sum_{n=1}^\infty a_n (2 - n^2 \pi^2 ) \cos (n \pi t)
\\
& = f(t)
=
\frac{1}{2} +
\sum_{\substack{n=1\\n~\text{odd}}}^\infty
\frac{-4}{\pi^2 n^2} \cos (n \pi t) .
\end{split}
\end{equation*}
Donc, $a_0 = \frac{1}{2}$, $a_n = 0$ pour $n$ paire ($n \geq 2$) et pour
$n$ impaire on a
\begin{equation*}
a_n (2 - n^2 \pi^2)
=
\frac{-4}{\pi^2 n^2} ,
\end{equation*}
ou
\begin{equation*}
a_n
=
\frac{-4}{n^2 \pi^2 (2 - n^2 \pi^2)} .
\end{equation*}
La série Fourier pour la solution $x(t)$ est
\begin{equation*}
x(t) = 
\frac{1}{4} +
\sum_{\substack{n=1\\n~\text{odd}}}^\infty
\frac{-4}{n^2 \pi^2 (2 - n^2 \pi^2)} 
\cos (n \pi t) .
\end{equation*}
\end{example}

\subsection{Exercises}

\begin{exercise}
Prenez $f(t) = {(t-1)}^2$ défini sur $0 \leq t \leq 1$.
\begin{tasks}
\task Esquissez le graphique de l'extension périodique paire de $ f $.
\task Esquissez le graphique de l'extension périodique impaire de $ f $.
\end{tasks}
\end{exercise}

\begin{exercise}
Trouvez la série de Fourier des extensions périodiques  paires impaires et paires
de la fonction $f(t) = {(t-1)}^2$ pour $0 \leq t \leq 1$.
Pouvez-vous dire quelle extension est continue à partir des coefficients de la série de Fourier?
\end{exercise}

\begin{exercise}
Trouvez la série de Fourier de l'extension périodique paire et impaire de
la fonction $f(t) = t$ pour $0 \leq t \leq \pi$.
\end{exercise}

\begin{exercise}
Trouvez la série de Fourier de l'extension périodique paire et impaire de
la fonction $f(t) = \sin t$ pour $0 \leq t \leq \pi$.
\end{exercise}

\begin{exercise}
\pagebreak[2]
Considerez
\begin{equation*}
x''(t) + 4 x(t) = f(t) ,
\end{equation*}
où $f(t) = 1$ sur $0 < t < 1$.
\begin{tasks}
\task Résolvez pour les conditions de Dirichlet $x(0)=0, x(1) = 0$.
\task Résolvez pour les Neumann  $x'(0)=0, x'(1) = 0$.
\end{tasks}
\end{exercise}

\begin{exercise}
Considerez
\begin{equation*}
x''(t) + 9 x(t) = f(t) ,
\end{equation*}
pour $f(t) = \sin (2\pi t)$ on $0 < t < 1$.
\begin{tasks}
\task Résolvez pour les conditions de Dirichlet  $x(0)=0, x(1) = 0$.
\task Résolvez pour les Neumann  $x'(0)=0, x'(1) = 0$.
\end{tasks}
\end{exercise}

\begin{exercise}
Considerez
\begin{equation*}
x''(t) + 3 x(t) = f(t) , \quad x(0) = 0, \quad x(1) = 0,
\end{equation*}
où $f(t) = \sum_{n=1}^\infty b_n \sin (n \pi t)$.  Écrivez la solution de $x(t)$
comme une série de Fourier où les coefficients sont donnés en terme de $b_n$.
\end{exercise}

\begin{exercise}
Soit $f(t) = t^2(2-t)$ pour $0 \leq t \leq 2$.  Soit $F(t)$ l'extension périodique impaire.  Calculez $F(1)$, $F(2)$, $F(3)$, $F(-1)$, $F(\nicefrac{9}{2})$,
$F(101)$, $F(103)$.  Note: Ne \textbf{pas} calculer la série sinusoïdale. 
\end{exercise}

\setcounter{exercise}{100}

\begin{exercise}
Soit $f(t) = \nicefrac{t}{3}$ sur $0 \leq t < 3$.
\begin{tasks}
\task Trouvez la série de Fourier de l'extension périodique paire.
\task Trouvez la série de Fourier de l'extension périodique impaire.
\end{tasks}
\end{exercise}
\exsol{%
a)
$\nicefrac{1}{2}
+
\sum\limits_{\substack{n=1\\n\text{ odd}}}^\infty
\frac{-4}{\pi^2 n^2}
\cos\bigl(\frac{n\pi}{3} t \bigr)$
\qquad
b) 
$\sum\limits_{n=1}^\infty
\frac{2{(-1)}^{n+1}}{\pi n}
\sin\bigl(\frac{n\pi}{3} t \bigr)$
}

\begin{exercise}
Soit $f(t) = \cos(2t)$ sur $0 \leq t < \pi$.
\begin{tasks}
\task Trouvez la série de Fourier de l'extension périodique paire.
\task Trouvez la série de Fourier de l'extension périodique impaire.
\end{tasks}
\end{exercise}
\exsol{%
a)
$\cos(2t)$
\qquad
b) 
$\sum\limits_{\substack{n=1 \\n \text{ odd}}}^\infty
\frac{-4n}{\pi n^2 - 4 \pi}
\sin(n t)$
}

\begin{exercise}
Soit $f(t)$ définie sur $0 \leq t < 1$.  Maintenant prenez
la moyenne des deux extensions
$g(t) = \frac{F_{\text{odd}}(t)+ F_{\text{even}}(t)}{2}$.
\begin{tasks}(2)
\task Qu'est-ce que $g(t)$ si $0 \leq t < 1$ (Justifiez!)
\task Qu'est-ce que $g(t)$ si $-1 < t < 0$ (Justifiez!)
\end{tasks}
\end{exercise}
\exsol{%
a) $f(t)$
\qquad
b) $0$
}

\begin{exercise}
Soit $f(t) = \sum_{n=1}^\infty \frac{1}{n^2} \sin(nt)$.  Résolvez
$x''- x = f(t)$ pour les conditions de Dirichlet $x(0) = 0$
et $x(\pi) = 0$.
\end{exercise}
\exsol{%
$\sum\limits_{n=1}^\infty \frac{-1}{n^2(1+n^2)} \sin(nt)$
}

\begin{exercise}[défi]
Soit $f(t) = t + \sum_{n=1}^\infty \frac{1}{2^n} \sin(nt)$.  Résolvez
$x'' + \pi x = f(t)$ pour les conditions de Dirichlet $x(0) = 0$
et $x(\pi) = 1$.  Astuce:  notez que $\frac{t}{\pi}$ satisfait aux conditions de Dirichlet données.
\end{exercise}
\exsol{%
$\frac{t}{\pi} + \sum\limits_{n=1}^\infty \frac{1}{2^n(\pi-n^2)} \sin(nt)$
}


%%%%%%%%%%%%%%%%%%%%%%%%%%%%%%%%%%%%%%%%%%%%%%%%%%%%%%%%%%%%%%%%%%%%%%%%%%%%%%

\sectionnewpage
\section{Applications des séries de Fourier}
\label{appoffourier:section}

%\sectionnotes{2 lectures\EPref{, \S9.4 dans \cite{EP}}\BDref{,
%pas dans \cite{BD}}}

\subsection{Oscillations périodiques forcées}

\begin{mywrapfigsimp}{2.0in}{2.3in}
\noindent
\inputpdft{massfigforce}
\end{mywrapfigsimp}
Revenons aux oscillations forcées. Considérons un système masse-ressort comme
avant, où nous avons une masse $ m $
sur un ressort ayant une constante de ressort $ k $,
avec un amortissement $ c $, et une force $ F (t) $ est appliquée à la masse. Supposons que
la fonction de forçage $ F (t) $ est $ 2L $ -périodique pour un certain $ L> 0 $.
Nous avons vu
ce problème dans le \chapterref{ho:chapter} avec $ F (t) = F_0 \cos (\omega t) $. L'équation qui régit cette configuration particulière est
\begin{equation} \label{afs:eq}
mx''(t) + cx'(t) + kx(t) = F(t) .
\end{equation}

La solution générale de \eqref{afs:eq} consiste en la solution complémentaire $ x_c $, qui
résout l'équation homogène associée $ mx '' + cx '+ kx = 0 $, et
une solution particulière de \eqref{afs:eq} qu'on appelle $ x_p $. Pour $ c> 0 $,
la solution complémentaire $ x_c $ est décroissante.
Par conséquent, on s'intéresse surtout à la solution particulière $ x_p $ qui est périodique avec la même période que $ F (t) $. On appelle cette solution particulière la
\emph{\myindex{solution périodique régulière}} et on l'écrit  $ x_ {sp} $ comme précédemment.
Ce qui est nouveau dans cette section est que nous considérons une fonction forcée arbitraire $ F (t) $ au lieu d'un simple cosinus.

Par souci de simplicité, supposons que $ c = 0 $. Le problème avec $ c> 0 $ est très
similaire.

L'équation
\begin{equation*}
mx'' + kx = 0 
\end{equation*}
a une solution générale
\begin{equation*}
x(t) = A \cos (\omega_0 t) + 
B \sin (\omega_0 t) ,
\end{equation*}
où $\omega_0 = \sqrt{\frac{k}{m}}$.
Toutes les solutions de
$mx''(t) + kx(t) = F(t)$ sont de la forme
$A \cos (\omega_0 t) + B \sin (\omega_0 t) + x_{sp}$.
La solution périodique constante $x_{sp}$ a la même période que  $F(t)$.

Dans l'esprit de la dernière section et avec l'idée des coefficients indéterminés
on écrit d'abord
\begin{equation*}
F(t) = \frac{c_0}{2} + \sum_{n=1}^\infty
c_n \cos \left( \frac{n \pi}{L} t \right) +
d_n \sin \left( \frac{n \pi}{L} t \right) .
\end{equation*}
Ensuite, on écrit une proposition de solution périodique constante $x$ telle que
\begin{equation*}
x(t) = \frac{a_0}{2} + \sum_{n=1}^\infty
a_n \cos \left( \frac{n \pi}{L} t \right) +
b_n \sin \left( \frac{n \pi}{L} t \right) ,
\end{equation*}
où $a_n$ et $b_n$ sont inconnus.
On remplace $x$ dans l'équation différentielle et l'on résout pour $a_n$ et pour
$b_n$ en terme de $c_n$ et $d_n$.  Ce processus
est souvent le mieux compris.
\pagebreak[2]

\begin{example} \label{afs:steadyex}
Supposons que $k=2$, et $m=1$.
Les unités sont à nouveau les unités mks\index{mks units}
(mètres-kilogrammes-secondes).
Il y a un jetpack attaché à la masse, qui tire avec une force de 1
newton pendant 1 seconde, puis s'éteint pendant 1 seconde, et ainsi de suite. On veut trouver la solution périodique constante.


L'équation est donc
\begin{equation*}
x'' + 2 x = F(t) ,
\end{equation*}
où $F(t)$ est la fonction intermédiaire
\begin{equation*}
F(t) =
\begin{cases}
0 & \text{if } \; {-1} < t < 0 , \\
1 & \text{if } \; \phantom{-}0 < t < 1 ,
\end{cases}
\end{equation*}
étend périodiquement.
On écrit
\begin{equation*}
F(t) = \frac{c_0}{2} + \sum_{n=1}^\infty
c_n \cos (n \pi t) +
d_n \sin (n \pi t) .
\end{equation*}
On calcule
\begin{align*}
c_n & = \int_{-1}^1 F(t) \cos (n \pi t) \, dt = 
\int_{0}^1 \cos (n \pi t) \, dt = 0 \qquad \text{for } \; n \geq 1,
\\
c_0 & = \int_{-1}^1 F(t) \, dt = 
\int_{0}^1 \, dt = 1 ,
\\
d_n & = \int_{-1}^1 F(t) \sin (n \pi t) \, dt
\\
& = \int_{0}^1 \sin (n \pi t) \, dt
\\
& = \left[ \frac{-\cos (n \pi t)}{n \pi} \right]_{t=0}^1
\\
& = \frac{1-{(-1)}^n}{\pi n} =
\begin{cases}
\frac{2}{\pi n} & \text{if } n \text{ impaire} , \\
0 & \text{if } n \text{ paire} .
\end{cases}
\end{align*}
Alors
\begin{equation*}
F(t) = \frac{1}{2} + \sum_{\substack{n=1 \\ n \text{ impaire}}}^\infty
\frac{2}{\pi n} \sin (n \pi t) .
\end{equation*}

On veut essayer
\begin{equation*}
x(t) = \frac{a_0}{2} + \sum_{n=1}^\infty
a_n \cos (n \pi t) +
b_n \sin (n \pi t) .
\end{equation*}
On remplace $x$ idans l'équation différentielle $x''+2x = F(t)$,
il est clair que $a_n = 0$ pour $n \geq 1$ car il n'y a pas de termes correspondants
dans la série pour
$F(t)$.  De manière similaire $b_n = 0$ pour $n$ pair.  Ainsi, on essaie
\begin{equation*}
x(t) = \frac{a_0}{2} +
\sum_{\substack{n=1 \\ n \text{ impaire}}}^\infty
b_n \sin (n \pi t) .
\end{equation*}
On remplace dans l'équation différentielle et l'on obtient 
\begin{equation*}
\begin{split}
x'' + 2 x & =
\sum_{\substack{n=1 \\ n \text{ impaire}}}^\infty
\Bigl[ - b_n n^2 \pi^2 \sin (n \pi t) \Bigr] + 
a_0 +
2
\sum_{\substack{n=1 \\ n \text{ impaire}}}^\infty
\Bigl[ b_n \sin (n \pi t) \Bigr]
\\
& =
a_0 +
\sum_{\substack{n=1 \\ n \text{ impaire}}}^\infty
b_n (2 - n^2 \pi^2 ) \sin (n \pi t)
\\
& =
F(t) = \frac{1}{2} + \sum_{\substack{n=1 \\ n \text{ impaire}}}^\infty
\frac{2}{\pi n} \sin (n \pi t) .
\end{split}
\end{equation*}
Alors $a_0 = \frac{1}{2}$, $b_n = 0$ pour un $n$ pair, et pour un $n$ impair, on obtient 
\begin{equation*}
b_n = 
\frac{2}{\pi n (2 - n^2 \pi^2 )} .
\end{equation*}

La solution périodique constant a la série de Fourier
\begin{equation*}
x_{sp}(t) = \frac{1}{4} + \sum_{\substack{n=1 \\ n \text{ odd}}}^\infty
\frac{2}{\pi n (2 - n^2 \pi^2 )}
\sin (n \pi t) .
\end{equation*}
On sait que c'est la solution périodique constante car elle ne contient aucun terme
de la solution complémentaire et elle est périodique avec la même période que
$ F (t) $ lui-même. Regardons \figurevref{afs:steadyexfig} pour le graphique de cette solution.
\begin{myfig}
\capstart
\diffyincludegraphics{width=3in}{width=4.5in}{afs-steadyex}
\caption{Graphe de la fonction périodique constante  $x_{sp}$, de l' 
\exampleref{afs:steadyex}.%
\label{afs:steadyexfig}}
\end{myfig}
\end{example}

\subsection{Résonance}

Tout comme lorsque la fonction de forçage était un simple cosinus, on peut rencontrer de la 
résonance. On suppose  que $ c = 0 $ et l'on discute seulement de la résonance pure.
Soit $ F (t) $ $ 2L $-périodique et considérons
\begin{equation*}
m x''(t) + k x (t) = F(t) .
\end{equation*}
Lorsqu'on développe $ F (t) $, on constate que certains de ses termes coïncident avec le
solution complémentaire à $ mx '' + kx = 0 $, on ne peut pas utiliser ces termes lorsqu'on
devine. Comme avant, ils disparaissent quand on les insère dans le terme de gauche et l'on obtient une équation contradictoire (telle que $ 0 = 1 $). On suppose que
\begin{equation*}
x_c = A \cos (\omega_0 t) + B \sin (\omega_0 t), 
\end{equation*}
où $\omega_0 = \frac{N \pi}{L}$ pour un certain entier positif $N$.
On doit modifier notre supposition et essayer  
\begin{equation*}
x(t) = \frac{a_0}{2} +
t \left(
a_N \cos \left( \frac{N \pi}{L} t \right) +
b_N \sin \left( \frac{N \pi}{L} t \right) \right) +
\sum_{\substack{n=1\\n\not= N}}^\infty
a_n \cos \left( \frac{n \pi}{L} t \right) +
b_n \sin \left( \frac{n \pi}{L} t \right) .
\end{equation*}
En d'autres mots, on multiplie le terme par $ t $. AÀpartir de là, on
procéde comme avant.

Bien sûr, la solution n'est pas une série de Fourier (ce n'est même pas
périodique), car elle contient les termes multipliés par $ t $. De plus, les
termes
$t \left( a_N \cos \left( \frac{N \pi}{L} t \right) +
b_N \sin \left( \frac{N \pi}{L} t \right) \right)$ dominent éventuellement et mènent à une oscillation intense. Comme précédemment, ce comportement est appelé une  \emph{\myindex{ résonance pure}} ou  \emph{\myindex{résonance}}.

On note qu'il peut maintenant y avoir une infinité de fréquences de résonance à atteindre.
Autrement dit, lorsqu'on change la fréquence de $ F $ (on change $ L $),  différents termes de la série de Fourier de $ F $ peuvent interférer avec la solution complémentaire et provoquer une résonance.
Cependant, il faut noter que puisque tout est une approximation et en
particulier $ c $ n'est jamais réellement zéro, mais quelque chose de très proche de zéro,
seulement la première fréquences de résonance compte dans la vraie vie.

\begin{example}
On veut résoudre l'équation
\begin{equation} \label{afs:eq-resonance}
2 x'' + 18 \pi^2 x = F(t) ,
\end{equation}
où
\begin{equation*}
F(t) =
\begin{cases}
-1 & \text{si } \; {-1} < t < 0 , \\
1 & \text{si } \; \phantom{-}0 < t < 1 ,
\end{cases}
\end{equation*}
étendue périodiquement.  On note que
\begin{equation*}
F(t) =
\sum_{\substack{n=1 \\ n \text{ impaire}}}^\infty
\frac{4}{\pi n}
\sin (n \pi t) . 
\end{equation*}

\begin{exercise}
Calculez la série de Fourier de $F$ pour vérifier l'équation ci-dessus.
\end{exercise}

Lorsque $\sqrt{\frac{k}{m}} = \sqrt{\frac{18\pi^2}{2}} = 3\pi$,
la solution de \eqref{afs:eq-resonance} est
\begin{equation*}
x(t) = c_1 \cos  (3\pi t) + c_2 \sin (3\pi t) + x_p (t)
\end{equation*}
pour une solution particulière $x_p$.

Si l'on essaie un $ x_p $ donné pour une série de Fourier avec $ \ sin (n \pi t) $ comme d'habitude,
l'équation complémentaire, $ 2x''  + 18 \pi ^2x = 0 $, mange  l'harmonique $3 ^ \text{e} $. Autrement dit, le terme
avec $ \sin (3 \pi t) $
est déjà dans dans la solution complémentaire.
Par conséquent, on retire ce terme et
on le multiplie par $ t $. On ajoute également un terme cosinus pour que tout soit correct.
Ainsi, on essaie
\begin{equation*}
x_p(t) =
a_3
t \cos (3 \pi t )
+
b_3
t \sin (3 \pi t)
+
\sum_{\substack{n=1 \\ n~\text{impaire} \\ n\not= 3}}^\infty
b_n
\sin (n \pi t) . 
\end{equation*}

Calculons la deuxième dérivée.
\begin{multline*}
x_p''(t) =
- 6 a_3
\pi \, \sin (3 \pi t) - 9\pi^2 a_3 \, t \, \cos (3 \pi t)
+
6 b_3
\pi \, \cos (3 \pi t) - 9\pi^2 b_3 \, t \, \sin (3 \pi t)
\\
{} +
\sum_{\substack{n=1 \\ n~\text{odd} \\ n\not= 3}}^\infty
(-n^2 \pi^2 b_n ) \,
\sin (n \pi t) . 
\end{multline*}
On remplace maintenant du côté gauche de l'équation différentielle.
\begin{align*}
2x_p'' + 18\pi^2 x_p 
= & 
- 12 a_3 \pi \sin (3 \pi t)
- 18\pi^2 a_3 t \cos (3 \pi t)
+ 12 b_3 \pi \cos (3 \pi t)
- 18\pi^2 b_3 t \sin (3 \pi t)
\\
& \phantom{\, - 12 a_3 \pi \sin (3 \pi t)} ~
{} + 18 \pi^2 a_3 t \cos (3 \pi t)
\phantom{\, + 12 b_3 \pi \cos (3 \pi t)} ~
{} + 18 \pi^2 b_3 t \sin (3 \pi t)
\\
& {} + \sum_{\substack{n=1 \\ n~\text{impaire} \\ n\not= 3}}^\infty
(-2n^2 \pi^2 b_n + 18\pi^2 b_n) \,
\sin (n \pi t) . 
\end{align*}
On simplifie
\begin{equation*}
2x_p'' + 18\pi^2 x_p =
- 12 a_3
\pi \sin (3 \pi t)
+
12 b_3
\pi \cos (3 \pi t)
+
\sum_{\substack{n=1 \\ n~\text{odd} \\ n\not= 3}}^\infty
(-2n^2 \pi^2 b_n + 18\pi^2 b_n)
\sin (n \pi t) . 
\end{equation*}
Cette série doit être égale à la série pour $F(t)$.
On résolve pour $a_3$ et pour $b_n$.
\begin{align*}
& a_3 = \frac{4/(3\pi)}{-12\pi} = \frac{-1}{9\pi^2} , \\
& b_3 = 0 , \\
& b_n = \frac{4}{n\pi(18\pi^2 - 2n^2 \pi^2)} 
= \frac{2}{\pi^3 n(9 - n^2)} \qquad \text{pour } n \text{ impaire et } n\not=3 .
\end{align*}

Ainsi,
\begin{equation*}
x_p(t) =
\frac{-1}{9\pi^2}
\,
t \, \cos (3 \pi t)
+
\sum_{\substack{n=1 \\ n~\text{odd} \\ n\not= 3}}^\infty
\frac{2}{\pi^3 n(9 - n^2)}
\sin (n \pi t) . 
\end{equation*}
\end{example}

Lorsque $ c> 0 $, on n'a pas à se soucier de la résonance pure. Il n'y a jamais de conflits et l'on n'a pas besoin de multiplier aucun terme par $ t $. Il existe un concept correspondant de \myindex{résonance pratique}
et c'est très similaire aux idées qu'on a déjà explorées dans
\chapterref{ho:chapter}.
Fondamentalement, ce qui se passe dans la résonance pratique est que l'un des
 coefficients de la série pour $ x_ {sp} $ peut devenir très gros. On ne va pas s'intéresser à ces détails ici.  

\subsection{Exercises}

\begin{exercise}
Soit $F(t) = \frac{1}{2} + \sum_{n=1}^\infty \frac{1}{n^2} \cos (n \pi t)$.
Trouvez la solution périodique régulière pour
$x'' + 2 x = F(t)$.  Exprimez votre solution sous la forme d'une série de Fourier. 
\end{exercise}

\begin{exercise}
Soit $F(t) = \sum_{n=1}^\infty \frac{1}{n^3} \sin (n \pi t)$.  Trouvez
la solution périodique constante de 
$x'' + x' + x = F(t)$. Exprimez votre solution sous la forme d'une série de Fourier. 
\end{exercise}

\begin{exercise}
Soit $F(t) = \sum_{n=1}^\infty \frac{1}{n^2} \cos (n \pi t)$. Trouvez
la solution périodique constante de 
$x'' + 4 x = F(t)$.  Exprimez votre solution sous la forme d'une série de Fourier. 
\end{exercise}

\begin{exercise}
Soit $F(t) = t$ pour $-1 < t < 1$ et étendez périodiquement.
Trouvez la solution périodique constante de 
$x'' + x = F(t)$.  Exprimez votre solution sous la forme d'une série.
\end{exercise}

\begin{exercise}
Soit $F(t) = t$ pour $-1 < t < 1$ et étendez périodiquement.
Trouvez la solution périodique constante de 
$x'' + \pi^2 x = F(t)$. Exprimez votre solution sous la forme d'une série.
\end{exercise}

\setcounter{exercise}{100}

\begin{exercise}
Soit $F(t) = \sin(2\pi t) + 0.1 \cos(10 \pi t)$.
Trouvez la solution périodique constante de $x'' + \sqrt{2}\, x = F(t)$.
Exprimez votre solution sous la forme d'une série de Fourier.
\end{exercise}
\exsol{%
$x = \frac{1}{\sqrt{2}-4 \pi^2} \sin(2\pi t) + \frac{0.1}{\sqrt{2}-100 \pi^2} \cos(10 \pi t)$
}

\begin{exercise}
Soit $F(t) = \sum_{n=1}^\infty e^{-n} \cos(2 n t)$.
Trouvez la solution périodique constante de $x'' + 3 x = F(t)$.
Exprimez votre solution sous la forme d'une série de Fourier.
\end{exercise}
\exsol{%
$x =
\sum\limits_{n=1}^\infty
\frac{e^{-n}}{3-{(2n)}^2} \cos(2n t)$
}

\begin{exercise}
Soit $F(t) = \lvert t \rvert$ pour $-1 \leq t \leq 1$ étendue périodiquement.
Trouvez la solution périodique constante de $x'' + \sqrt{3}\, x = F(t)$.
Exprimez votre solution sous la forme d'une série de Fourier.
\end{exercise}
\exsol{%
$x =
\frac{1}{2\sqrt{3}} + 
\sum\limits_{\substack{n=1 \\ n \text{ impaire}}}^\infty \frac{-4}{n^2 \pi^2
(\sqrt{3}-n^2 \pi^2)} \cos (n \pi t)$
}

\begin{exercise}
Soit $F(t) = \lvert t \rvert$ pour $-1 \leq t \leq 1$ étendue périodiquement.
Trouvez la solution périodique constante de $x'' + \pi^2 x = F(t)$.
Exprimez votre solution sous la forme d'une série de Fourier.
\end{exercise}
\exsol{%
$x =
\frac{1}{2\sqrt{3}} -
\frac{2}{\pi^3} t \sin(\pi t) + 
\sum\limits_{\substack{n=3 \\ n \text{ impaire}}}^\infty \frac{-4}{n^2 \pi^4
(1-n^2)} \cos (n \pi t)$
}


%%%%%%%%%%%%%%%%%%%%%%%%%%%%%%%%%%%%%%%%%%%%%%%%%%%%%%%%%%%%%%%%%%%%%%%%%%%%%%

\sectionnewpage
\section{Équation différentielle partielle, separation de variables, et l'équation de la chaleur}
\label{heateq:section}

\sectionnotes{2 lectures\EPref{, \S9.5 dans \cite{EP}}\BDref{,
\S10.5 dans \cite{BD}}}

Rappelons qu'une \emph{\myindex{équation différentielle partielle}} ou
\emph{\myindex{EDP}} est une équation contenant les dérivées partielles
par rapport à \emph{plusieurs} variables indépendantes. La résolution des EDP
est la  principale application de la série Fourier que l'on s'intéressera.

Une EDP est dite  \emph{linéaire\index{linear PDE}} si la variable dépendante et sa dérivée apparaissent au moins à la première puissance et dans aucune des fonctions. On dira que c'est une EDP linéaire. Avec la EDP\@,
on spécifie généralement quelques
\emph{conditions aux frontières \index{boundary conditions for a PDE}},
où la valeur de la solution ou de ses dérivés est donnée par
la limite d'une région, et / ou de
certaines
\emph{conditions initiales \index{initial conditions for a PDE}} 
où la valeur
de la solution ou de ses dérivés est donnée à un certain temps initial.
Parfois, de telles conditions sont mélangées et l'on y fait référence
simplement comme
\emph{conditions aux bord \index{side conditions for a PDE}}.

Nous allons étudier ici l'
\emph{\myindex{équation de la chaleur}}, qui est un exemple de \emph{\myindex{EDP parabolique }}.  

Ensuite, au tour de l'
\emph{\myindex{équation de l'onde}}, qui est un exemple de \emph{\myindex{EDP hyperbolique}}.  
%Finalement, on étudiera l'
%\emph{\myindex{équation de Laplace }}, qui est un exemple de  %\emph{\myindex{EDP elliptique }}.  
%Chacun des exemples illustrera le
*%omportement typique de toute la classe.

\subsection{Chaleur sur un fil isolé}

Commençons par l'équation de la chaleur.
Considérons un fil (ou une fine tige métallique) de longueur $ L $
qui est isolé sauf au point de terminaison.  Soit $ x $ la position le long du fil et soit $ t $ le temps.  Regardons la \figurevref{heat:wirefig}.

\begin{myfig}
\capstart
\inputpdft{heat-wire}
\caption{Fil isolé.\label{heat:wirefig}}
\end{myfig}

Soit $u(x,t)$ la température au point $ x $ au temps $ t $.
L'équation régissant cette configuration se nomme l' \emph{\myindex{équation de chaleur à une dimension}}\index{heat equation}:
\begin{equation*}
\mybxbg{~~
\frac{\partial u}{\partial t} =
k \frac{\partial^2 u}{\partial x^2} ,
~~}
\end{equation*}
où $k > 0$ est la constante (the \emph{\myindex{de conductivité thermale}} du matériel).
Ainsi, le changement de chaleur en un point spécifique est proportionnel à la seconde
dérivé de la chaleur le long du fil.  C'est logique puisqu'en effet,
si à un $ t $ fixe le graphique de la répartition de la chaleur a un maximum (le graphique est concave vers le bas),
puis la chaleur s'échappe du maximum.  Et vice versa.

On utilise généralement une notation plus pratique pour les dérivées partielles.
On écrit $ u_t $ au lieu de $\frac{\partial u}{\partial t}$,
et l'on écrit $u_{xx}$ plutôt que $\frac{\partial^2 u}{\partial x^2}$.
Avec cette notation,  l'équation de la chaleur devient 
\begin{equation*}
u_t = k u_{xx} .
\end{equation*}

Pour l'équation de la chaleur, il faut aussi avoir quelques
conditions aux limites.
On suppose que les extrémités du fil sont soit exposées
et touchent à un corps ayant une chaleur constante,  ou  soit les extrémités sont isolées.
Si les extrémités du fil, elles sont maintenues à la température 0, alors
les conditions sont
\begin{equation*}
u(0,t) = 0 \qquad \text{et} \qquad u(L,t) = 0.
\end{equation*}
Si,  d'autre part,  les extrémités sont également isolées,  les conditions sont
\begin{equation*}
u_x(0,t) = 0 \qquad \text{et} \qquad
u_x(L,t) = 0 .
\end{equation*}
Regardons pourquoi c'est ainsi.
Si $ u_x $ est positif à un certain moment $ x_0 $,  alors à un certain moment ,
$ u $ est plus petit à gauche de $ x_0 $ et plus grand à droite de $ x_0 $.
La chaleur passe d'une température élevée à une température basse, c'est-à-dire vers la gauche.
Par contre,  si $ u_x $ est négatif, la chaleur circule à nouveau
du feu vif au feu doux, c'est-à-dire à droite.  Donc,  quand $ u_x $ est nul,  c'est un point où la chaleur n'est pas conduite.  En d'autres termes, $ u_x (0, t) = 0 $ signifie qu'aucune chaleur ne circule dans ou hors du fil au point $ x = 0 $.

On a deux conditions le long de l'axe $ x $, car il y a
deux dérivés dans la direction $ x $.
On dit que ces conditions secondaires sont
\emph{homogènes \index{homogeneous side conditions}}
(i.e.,  que $u$ ou la dérivée de $u$ est nulle).

On a également besoin d'une condition initiale --- la distribution de la température
au temps $ t = 0 $.  C'est,
\begin{equation*}
u(x,0) = f(x) ,
\end{equation*}
pour une certaine fonction connue $ f (x) $.
Cette condition initiale n'est pas une condition homogène.

\subsection{Séparation de variable}

L'équation de la chaleur est linéaire comme $ u $ et ses dérivés n'apparaissent à toutes les puissances ou dans toutes les fonctions.
Ainsi,  le principe de \myindex{superposition} s'applique toujours pour
l'équation de la chaleur
(sans condition secondaire):
Si $ u_1 $ et $ u_2 $ sont des
solutions et si $ c_1 $, et $ c_2 $ sont des constantes,  alors
$ u = c_1 u_1 + c_2 u_2 $ est aussi une solution.

\begin{exercise}
Vérifiez le principe de superposition de l'équation de chaleur.
\end{exercise}

La superposition préserve certaines des conditions secondaires.  En particulier,
si $ u_1 $ et $ u_2 $ sont des 
solutions qui satisfont à $ u (0, t) = 0 $ et à $ u (L, t) = 0 $,
et que $ c_1 $  et $ c_2 $ sont des constantes,  alors
$ u = c_1 u_1 + c_2 u_2 $ est toujours une solution
qui satisfait à $ u (0, t) = 0 $ et à  $ u (L, t) = 0 $.  Il en est de même
pour les conditions secondaires $ u_x (0, t) = 0 $ et $ u_x (L, t) = 0 $.  En général,
la superposition préserve toutes les conditions homogènes.

La méthode de 
\emph{séparation de variable\index{separation of variables}} consiste à essayer de trouver des solutions qui sont des produits de fonctions à une variable.
Pour l'équation de la chaleur,  on essaye de trouver des solutions de la forme
\begin{equation*}
u(x , t) = X(x)T(t) .
\end{equation*}
Estpérer que la solution souhaitée soit de cette forme est trop
espérer.  Ce qui est parfaitement raisonnable de demander,  cependant,  c'est de trouver
assez de \myquote{blocs de construction} de solutions de la forme
$u(x,t) = X(x)T(t)$.  En utilisant cette procédure
de sorte que la solution souhaitée à la EDP soit en quelque sorte construite à partir de ces
blocs de constructions par l'utilisation de la superposition. 

Essayons de résoudre l'équation de la chaleur
\begin{equation*}
u_t = k u_{xx}
\qquad \text{with} \quad
u(0 , t) = 0 ,\quad \quad u(L , t) = 0,
\quad \text{et} \quad u(x,0) = f(x) .
\end{equation*}
On suppose que $u(x,t) = X(x)T(t)$.  On essaie de faire en sorte que cette hypothèse satisfasse à l'équation différentielle, $ u_t = k u_ {xx} $, et les conditions homogènes,
$ u (0, t) = 0 $ et $ u (L, t) = 0 $.  Ensuite,  comme la superposition préserve l'équation différentielle et les conditions homogènes secondaires,  on essaie de
construire une solution à partir de ces éléments de base pour résoudre les
conditions initiales non homogènes $ u (x, 0) = f (x) $.

D'abord, on remplace $u(x, t) = X(x)T(t)$ dans l'équation de la chaleur et l'on obtient
\begin{equation*}
X(x)T'(t) = k X''(x)T(t) .
\end{equation*}
On réécrit comme
\begin{equation*}
\frac{T'(t)}{k T(t)} =
\frac{X''(x)}{X(x)} .
\end{equation*}
Cette équation doit être valable pour tout $ x $ et pour tout $ t $.  Mais le
le côté gauche ne dépend pas de $ x $ et le côté droit ne dépend pas de
de $ t $.  Par conséquent,  chaque côté doit être une constante.  Appelons cette
constante $-\lambda$ (le signe moins est pour faciliter le travail plus tard).
On obtient les deux équations suivantes
\begin{equation*}
\frac{T'(t)}{k T(t)} = -\lambda =
\frac{X''(x)}{X(x)} .
\end{equation*}
En d'autres mots,
\begin{align*}
X''(x) + \lambda X(x) &= 0 , \\
T'(t) + \lambda k T(t) &= 0 .
\end{align*}
La condition aux limites $ u (0, t) = 0 $ implique $ X (0) T (t) = 0 $.  On cherche
une solution non triviale et l'on peut supposer que $ T (t) $ n'est pas identiquement nul.  D'où $ X (0) = 0 $.  De même, $ u (L, t) = 0 $ implique $ X (L) = 0 $.  On
recherche des solutions non triviales $ X $ du problème de valeurs propres
$X'' + \lambda X = 0$, $X(0) = 0$, $X(L) = 0$.  
On a déjà constaté que les seules valeurs propres sont $\lambda_n = \frac{n^2 \pi^2}{L^2}$,  pour les entiers 
$n \geq 1$,
où les valeurs propres sont $\sin \left(\frac{n \pi}{L} x\right)$.  Par conséquent, choisissons
les solutions
\begin{equation*}
X_n (x) = \sin \left(\frac{n \pi}{L} x \right) .
\end{equation*}
Le $ T_n $ correspondant doit satisfaire à l'équation
\begin{equation*}
T_n'(t) + \frac{n^2 \pi^2}{L^2} k T_n(t) = 0 .
\end{equation*}
C'est l'une des
\hyperref[subsection:fourfundamental]{équations fondamentales},
et la solution est une exponentielle:
\begin{equation*}
T_n(t) = e^{\frac{-n^2 \pi^2}{L^2} k t} .
\end{equation*}
Il est utile de noter que $T_n(0) = 1$.
Les blocs de constructions de la solution sont
\begin{equation*}
u_n(x,t) = X_n(x)T_n(t) =
\sin \left( \frac{n \pi}{L} x \right)
e^{\frac{-n^2 \pi^2}{L^2} k t} .
\end{equation*}

On note que $u_n(x,0) = \sin \left( \frac{n \pi}{L} x \right)$.  Écrivons $f(x)$ comme une série sinusoïdale
\begin{equation*}
f(x) = \sum_{n=1}^\infty b_n \sin \left(\frac{n \pi}{L}  x \right) .
\end{equation*}
Autrement dit,  on trouve la série de Fourier de l'extension périodique impaire de $ f (x) $.
On a utilisé la série sinus,  car elle correspond au problème des valeurs propres pour
$ X (x) $ ci-dessus.
Enfin,  on utilise la superposition pour écrire la solution comme
\begin{equation*}
\mybxbg{~~
u(x,t) = 
\sum_{n=1}^\infty
b_n
u_n(x,t)
=
\sum_{n=1}^\infty
b_n
\sin \left( \frac{n \pi}{L}  x \right)
e^{\frac{-n^2 \pi^2}{L^2} k t} .
~~}
\end{equation*}

Pourquoi cette solution fonctionne-t-elle? Notez tout d'abord qu'il s'agit d'une solution de
l'équation de la chaleur par superposition.  Elle satisfait à $ u (0, t) = 0 $
et à $ u (L, t) = 0 $,  car $ x = 0 $ ou $ x = L $ fait disparaître tous les sinus.
Enfin,  en branchant $ t = 0 $,  on remarque que $ T_n (0) = 1 $ et ainsi
\begin{equation*}
u(x,0) = 
\sum_{n=1}^\infty
b_n
u_n(x,0)
=
\sum_{n=1}^\infty
b_n
\sin \left( \frac{n \pi}{L} x \right)
=
f(x) .
\end{equation*}

\begin{example}

Considérons un fil isolé de longueur 1 dont
les extrémités sont noyées dans la glace (température 0).
Soit $ k = 0,003 $ et soit la distribution de chaleur initiale $ u (x, 0) = 50 \, x \, (1-x) $.
Regardons  \figurevref{heat:wireexinitfig}.
On suppose qu'on veuille trouver la fonction de température $ u (x, t) $.  On suppose aussi qu'on veuille trouver quand (à quoi $ t $) fait la température maximale dans le fil
est la moitié du maximum initial de 12,5.

\begin{myfig}
\capstart
\diffyincludegraphics{width=3in}{width=4.5in}{heat-wireex-init}
\caption{Distribution initiale de la température dans le
câble.\label{heat:wireexinitfig}}
\end{myfig}

On résout le problème d'EDP suivant: 
\begin{align*}
& u_t = 0.003 \, u_{xx} , \\
& u(0,t) = u(1,t) = 0 , \\
& u(x,0) = 50\,x\,(1-x) \qquad \text{pour } \; 0 < x < 1 .
\end{align*}
On écrit $f(x) = 50\,x\,(1-x)$ pour $0 < x < 1$ comme une série de sinus.  Alors 
$
f(x) = \sum_{n=1}^\infty b_n \sin (n \pi x) ,
$
où
\begin{equation*}
b_n = 2 \int_0^1 50\,x\,(1-x) \sin (n \pi x) \,dx
= 
\frac{200}{{\pi }^{3}{n}^{3}}-\frac{200\,{\left( -1\right) }^{n}}{{\pi }^{3}{n}^{3}}
=
\begin{cases}
0 & \text{si } n \text{ paire} , \\
\frac{400}{\pi^3 n^3} & \text{si } n \text{ impaire} .
\end{cases}
\end{equation*}

La solution $u(x,t)$,  dans le graphique
\figurevref{heat:wireexfig} pour $0 \leq t \leq 100$,
est donnée par la série:
\begin{equation*}
u(x,t) = 
\sum_{\substack{n=1 \\ n \text{ odd}}}^\infty
\frac{400}{\pi^3 n^3}
\sin (n \pi x )
\, e^{-n^2 \pi^2 \, 0.003 \, t} .
\end{equation*}

\begin{myfig}
\capstart
\diffyincludegraphics{width=5in}{width=7.5in}{heat-wireex}
\caption{Graphie de la température dans le câble ua point $x$
et au temps $t$.\label{heat:wireexfig}}
\end{myfig}

Enfin,  on répond à la question de la température maximale.  Il est
relativement facile de voir
que la température maximale à tout moment fixe est toujours $ x = 0,5 $ dans ,
le milieu du fil.  Le graphique de $ u (x, t) $ confirme cette intuition.
Si l'on branche $ x = 0,5 $,  on obtient
\begin{equation*}
u(0.5,t) = 
\sum_{\substack{n=1 \\ n \text{ odd}}}^\infty
\frac{400}{\pi^3 n^3}
\sin (n \pi\, 0.5 )
\, e^{-n^2 \pi^2 \, 0.003 \, t} .
\end{equation*}

Pour $n=3$ et plus grand (on re rappelle que $n$ est seulement impaire),  les termes
de la série
sont insignifiants par rapport au premier terme.
Le premier terme de la série est déjà une très bonne approximation
de la fonction.
Par conséquent,
\begin{equation*}
u(0.5,t) \approx
\frac{400}{\pi^3}
\, e^{-\pi^2 \, 0.003 \, t} .
\end{equation*}
L'approximation s'améliore à mesure que $ t $ grandit et que les autres
termes se dégradent de plus en plus rapidement.
On trace la fonction $ u (0.5, t) $,  la température au milieu du fil
au temps $ t $,  en \figurevref{heat:wireexmaxfig}.  La figure aussi
représente l'approximation pour le premier terme.

\begin{myfig}
\capstart
\diffyincludegraphics{width=3in}{width=4.5in}{heat-wireex-max}
\caption{Température au milieu du fil (la courbe du bas),
et l'approximation de cette température en n'utilisant que le premier terme de
la série (courbe du haut). \label{heat:wireexmaxfig}}
\end{myfig}

Après $ t = 5 $ environ
il serait difficile de faire la différence
entre le premier terme de la série pour $ u (x, t) $ et
la vraie solution $ u (x, t) $.  Ce comportement
est une caractéristique générale de la résolution de l'équation de la chaleur.
Si vous êtes intéressé par un comportement pour des $ t $ assez grands,  seul le
premier ou les deux premiers termes peuvent être nécessaires.


Revenons à la question qui cherche à savoir quand est-ce que la température maximale est égale à la moitié de la
température maximale initiale.  Autrement dit,  quand la température
au milieu $\nicefrac{12.5}{2} = 6.25$.  On remarque sur le graphique que si l'on utilise
l'approximation par le premier terme on sera assez proche.  On résout 
\begin{equation*}
6.25 =
\frac{400}{\pi^3}
\, e^{-\pi^2 \, 0.003 \, t} .
\end{equation*}
Alors,
\begin{equation*}
t =
\frac{\ln \frac{6.25\,\pi^3}{400}}{-\pi^2 0.003}
\approx 24.5 .
\end{equation*}
Ainsi, la température maximale tombe de moitié à environ à $t=24.5$.
\end{example}

On mentionne un comportement intéressant de la solution à l'équation de la chaleur.
L'équation de la chaleur est
\myquote{lisse} en dehors de la fonction $ f (x) $ au fur et à mesure que $ t $ grandit.  Pour un $ t $ fixe,
la solution est une série de Fourier à coefficients
$b_n e^{\frac{-n^2 \pi^2}{L^2} k t}$.  Si $t > 0$,  puis ces coefficients
deviennent nuls plus vite que tout $ \ frac {1} {n ^ p} $ pour toute puissance $ p $.  En d'autre
mots,  la série de Fourier a une infinité de dérivés partout .
Ainsi,  même si la fonction $ f (x) $ fait des sauts et des coins,  alors pour
un  $ t> 0 $ fixe,  la solution
$ u (x, t) $ en fonction de $ x $ est aussi lisse que possible.

\begin{example}
Lorsque la condition initiale est déjà une série sinusoïdale,  alors on n'a pas besoin
pour calculer quoi que ce soit; il vous suffit de remplacer.
\begin{equation*}
u_t = 0.3 \, u_{xx}, \qquad u(0,t)=u(1,t)=0, \qquad u(x,0) = 0.1 \sin(\pi t) +
\sin(2\pi t) .
\end{equation*}
La solution est alors
\begin{equation*}
u(x,t) =
0.1 \sin(\pi t) e^{- 0.3 \pi^2 t}
+ 
\sin(2 \pi t) e^{- 1.2 \pi^2 t} .
\end{equation*}
\end{example}

\subsection{Extrémités isolées}

Supposons maintenant que les extrémités du fil soient isolées.  Dans ce cas,  on résolve
l'équation
\begin{equation*}
u_t = k u_{xx}
\qquad \text{avec} \quad
u_x(0,t) = 0, \quad u_x(L,t) = 0,
\quad \text{et} \quad u(x,0) = f(x) .
\end{equation*}
Encore une fois,  on essaye une solution de la forme $ u (x, t) = X (x) T (t) $.  On utilise la même
procédure qu'avant pour remplacer dans l'équation de la chaleur et l'on arrive aux
deux équations suivantes
\begin{align*}
X''(x) + \lambda X(x) &= 0 , \\
T'(t) + \lambda k T(t) &= 0 .
\end{align*}

À ce stade, l'histoire change légèrement.
La condition aux limites $u_x(0,t) = 0$ implique  $X'(0)T(t) = 0$.
Ainsi $X'(0) = 0$.  De manière similaire,  $u_x(L,t) = 0$ implique $X'(L) = 0$.  On
recherche des solutions non triviales $ X $ au problème de valeurs propres
$X'' + \lambda X = 0$, $X'(0) = 0$, $X'(L) = 0$.  On a déjà constaté que
les seules valeurs propres sont $\lambda_n = \frac{n^2 \pi^2}{L^2}$,  pour les entiers
$n \geq 0$,
où les fonctions propres sont $\cos \left( \frac{n \pi}{L} x\right)$
(on inclut la constante
fonction propre).  Par conséquent,  choisissons
solutions
\begin{equation*}
X_n (x) = \cos \left( \frac{n \pi}{L} x \right)
\qquad \text{et} \qquad
X_0 (x) = 1.
\end{equation*}
Le $ T_n $ correspondant doit satisfaire à l'équation
\begin{equation*}
T_n'(t) + \frac{n^2 \pi^2}{L^2} k T_n(t) = 0 .
\end{equation*}
Pour $n \geq 1$,  comme précédemment,
\begin{equation*}
T_n(t) = e^{\frac{-n^2 \pi^2}{L^2} k t} .
\end{equation*}
Pour $n = 0$,  on a $T_0'(t) = 0$ et alors $T_0(t) = 1$.
Nos solutions de base sont
\begin{equation*}
u_n(x,t) = X_n(x)T_n(t) =
\cos \left( \frac{n \pi}{L} x \right)
e^{\frac{-n^2 \pi^2}{L^2} k t} ,
\end{equation*}
et
\begin{equation*}
u_0(x,t) = 1 .
\end{equation*}

On note que $u_n(x,0) = \cos \left( \frac{n \pi}{L} x \right)$.  On écrit $f$ en utilisant une série cosinus 
\begin{equation*}
f(x) = \frac{a_0}{2} + \sum_{n=1}^\infty a_n \cos \left( \frac{n \pi}{L} x
\right) .
\end{equation*}
Autrement dit,  on trouve la série de Fourier de l'extension  périodique de $f(x)$.

On utilise la superposition pour écrire la solution comme
\begin{equation*}
\mybxbg{~~
u(x,t) = 
\frac{a_0}{2} + 
\sum_{n=1}^\infty
a_n
u_n(x,t)
=
\frac{a_0}{2} + 
\sum_{n=1}^\infty
a_n
\cos \left( \frac{n \pi}{L} x \right)
e^{\frac{-n^2 \pi^2}{L^2} k t} .
~~}
\end{equation*}

\begin{example}
Essayons la même équation que précédemment,  mais pour des extrémités isolées.
On résolve le problème d'EDPsuivant
\begin{align*}
& u_t = 0.003 \, u_{xx} , \\
& u_x(0,t) = u_x(1,t) = 0 , \\
& u(x,0) = 50\,x\,(1-x) \qquad \text{pour } \; 0 < x < 1 .
\end{align*}

Pour ce problème,  on doit trouver la série cosinus de $ u (x, 0) $.
Pour 0 $ <x <1 $ on a
\begin{equation*}
50\, x\,(1-x)
=
\frac{25}{3} +
\sum_{\substack{n=2 \\ n \text{ paire}}}^\infty
\left( \frac{-200}{\pi^2 n^2} \right)
\cos (n \pi x) .
\end{equation*}
Le calcul est laissé au lecteur.
Par conséquent,  la solution au problème d'EDP tracée dans
\figurevref{heat:wireisolexfig},  est donnée par la série
\begin{equation*}
u(x,t)
=
\frac{25}{3} +
\sum_{\substack{n=2 \\ n \text{ paire}}}^\infty
\left( \frac{-200}{\pi^2 n^2} \right)
\cos ( n \pi x)
\, e^{-n^2 \pi^2 \, 0.003 \, t} .
\end{equation*}

\begin{myfig}
\capstart
\diffyincludegraphics{width=5in}{width=7.5in}{heat-wireisolex}
\caption{Tracé de la température du fil isolé à la position $ x $ et
au temps $ t $. \label{heat:wireisolexfig}}
\end{myfig}

On remarque dans le graphique
qu'au fil du temps,  la température s'équilibre à travers le fil.  Finalement,  tous les
termes sauf la constante
morte,  et l'on a une température uniforme
 $\frac{25}{3} \approx 8.33$ sur toute la longueur du fil.
\end{example}

On développe sur le dernier point.  Le terme constant dans la série est
\begin{equation*}
\frac{a_0}{2} = \frac{1}{L} \int_0^L f(x) \, dx .
\end{equation*}
En d'autres mots,  $\frac{a_0}{2}$ est la valeur moyenne de $f(x)$, ce qui est,
la moyenne de la température initiale.  Comme le fil est isolé
partout,  aucune chaleur ne peut sortir et aucune chaleur ne peut entrer.  Donc la température
essaie de se répartir uniformément dans le temps,  et la température moyenne doit toujours être la
pareille, en particulier elle est toujours égale à $\frac{a_0}{2}$.  
Lorsque le temps tend vers l'infini,  la température devient égale à la constante $\frac{a_0}{2}$ partout.

\subsection{Exercises}

\begin{exercise}
Considérez un fil de longueur 2,  avec $ k = 0,001 $ et une 
distribution initiale de température $ u (x, 0) = 50 x $.  Les deux extrémités
sont noyés dans la glace (température 0).  Trouvez la solution sous la forme d'une série.
\end{exercise}

\begin{exercise}
Trouvez une solution à la série 
\begin{align*}
& u_t =  u_{xx} , \\
& u(0,t) = u(1,t) = 0 , \\
& u(x,0) = 100 \qquad \text{pour } \; 0 < x < 1 .
\end{align*}
\end{exercise}

\begin{exercise}
Trouvez une solution à la série 
\begin{align*}
& u_t =  u_{xx} , \\
& u_x(0,t) = u_x(\pi,t) = 0 , \\
& u(x,0) = 3\cos (x) + \cos (3x) \qquad \text{pour } \; 0 < x < \pi .
\end{align*}
\end{exercise}

\begin{exercise} \label{heat:cosexr}
Trouvez une solution à la série 
\begin{align*}
& u_t = \frac{1}{3} u_{xx} , \\
& u_x(0,t) = u_x(\pi,t) = 0 , \\
& u(x,0) = \frac{10x}{\pi} \qquad \text{pour } \; 0 < x < \pi .
\end{align*}
\end{exercise}

\begin{exercise} \label{heat:oneto100exr}
Trouvez une solution à la série 
\begin{align*}
& u_t =  u_{xx} , \\
& u(0, t) = 0 , \quad u(1,t) = 100 , \\
& u(x, 0) = \sin (\pi x) \qquad \text{pour } \; 0 < x < 1 .
\end{align*}
Astuce: Utilisez le fait que $u(x, t) = 100 x$ est une solution satisfaisant à 
$u_t = u_{xx}$, $u(0, t) = 0$, $u(1, t) = 100$.  Ensuite, utilise la superposition.
\end{exercise}

\begin{exercise}
Trouvez la \emph{\myindex{température constante}} de la solution comme une fonction de $x$ uniquement,
en laissant $t \to
\infty$ dans la solution de l'exercie \ref{heat:cosexr} et \ref{heat:oneto100exr}.
Vérifiez qu'elle satisfait à l'équation $u_{xx} = 0$.
\end{exercise}

\begin{exercise}
Utilisez la séparation de variables pour trouver un élément non trivial à la
solution  $ u_ {xx} + u_ {yy} = 0 $,  où $ u (x, 0) = 0 $ et $ u (0, y) = 0 $.
Astuce: Essayez$u(x,y) = X(x)Y(y)$.
\end{exercise}

\begin{exercise}[challenging]
Supposez qu'une extrémité du fil soit isolée (disons à $ x = 0 $) et
l'autre extrémité soit maintenue à température nulle.  C'est-à-dire,
trouvez une solution en série de
\begin{align*}
& u_t = k u_{xx} , \\
& u_x(0,t) = u(L,t) = 0 , \\
& u(x,0) = f(x) \qquad \text{pour } \; 0 < x < L .
\end{align*}
Exprimez tous les coefficients de la série par des intégrales de $f(x)$.
\end{exercise}

\begin{exercise}[challenging]
Supposez que le fil soit circulaire et isolé,  donc qu'il n'y a pas d'extrémité.
Vous pouvez considérer cela comme une simple connexion des deux extrémités puis
en vous assurant que la solution corresponde.
Autrement dit,  trouvez une solution en série de
\begin{align*}
& u_t = k u_{xx} , \\
& u(0,t) = u(L,t) , \qquad
u_x(0,t) = u_x(L,t) , \\
& u(x,0) = f(x) \qquad \text{pour } \; 0 < x < L .
\end{align*}
Exprimez tous les coefficients de la série par des intégrales de  $f(x)$.
\end{exercise}

\begin{exercise}
Considérez un fil isolé aux deux extrémités,  $L=1$, $k=1$,
et $u(x,0) = \cos^2(\pi x)$.
\begin{tasks}
\task
Trouvez la solution de $u(x,t)$.  Astuce: Une identité trigonométrique.
\task
Trouvez la température moyenne.
\task
Au départ,  la variation de température est de 1 (le maximum moins le minimum).
Trouvez l'heure à laquelle la variation est $\nicefrac{1}{2}$.
\end{tasks}
\end{exercise}

\setcounter{exercise}{100}

%u(x,t) = 
%\sum_{n=1}^\infty
%b_n
%u_n(x,t)
%=
%\sum_{n=1}^\infty
%b_n
%\sin \left( \frac{n \pi}{L} x \right)
%e^{\frac{-n^2 \pi^2}{L^2} k t} .

\begin{exercise}
Find a series solution of
\begin{align*}
& u_t =  3 u_{xx} , \\
& u(0,t) = u(\pi,t) = 0 , \\
& u(x,0) = 5\sin (x) + 2\sin (5x) \qquad \text{for } \; 0 < x < \pi .
\end{align*}
\end{exercise}
\exsol{%
$u(x,t) = 
5
\sin (x)
\, e^{- 3 t}
+
2
\sin (5x)
\, e^{-75 t}$
}

%u(x,t) = 
%\frac{a_0}{2} + 
%\sum_{n=1}^\infty
%a_n
%u_n(x,t)
%=
%\frac{a_0}{2} + 
%\sum_{n=1}^\infty
%a_n
%\cos \left( \frac{n \pi}{L} x \right)
%\, e^{\frac{-n^2 \pi^2}{L^2} k t} .

\begin{exercise}
Find a series solution of
\begin{align*}
& u_t =  0.1 u_{xx} , \\
& u_x(0,t) = u_x(\pi,t) = 0 , \\
& u(x,0) = 1 + 2\cos (x) \qquad \text{for } \; 0 < x < \pi .
\end{align*}
\end{exercise}
\exsol{%
$u(x,t) = 
1 + 
2
\cos (x)
\, e^{-0.1 t}$
}

\begin{exercise}
Use separation of variables to find a nontrivial solution to
$u_{xt} = u_{xx}$.
\end{exercise}
\exsol{%
$u(x,t) = e^{\lambda t} e^{\lambda x}$ for some $\lambda$
}

\begin{exercise}
Use separation of variables (Hint: try $u(x,t) = X(x)+T(t)$)
to find a nontrivial solution to
$u_{x} + u_{t} = u$.
\end{exercise}
\exsol{%
$u(x,t) = Ae^x + Be^t$
}

\begin{exercise}
Suppose that the temperature on the wire is fixed at $0$
at the ends, $L=1$, $k=1$, and $u(x,0) = 100\sin(2 \pi x)$.
\begin{tasks}
\task
What is the temperature at $x = \nicefrac{1}{2}$ at any time.
\task
What is the maximum and the minimum temperature on the wire
at $t=0$.
\task
At what time is the maximum temperature on the wire exactly
one half of the initial maximum at $t=0$.
\end{tasks}
\end{exercise}
\exsol{%
a) $0$,
\quad b) minimum $-100$, maximum $100$,
\quad c) $t = \frac{\ln 2}{4 \pi^2}$.
}

%%%%%%%%%%%%%%%%%%%%%%%%%%%%%%%%%%%%%%%%%%%%%%%%%%%%%%%%%%%%%%%%%%%%%%%%%%%%%%

\sectionnewpage
\section{Équation d'onde unidimensionnelle} \label{we:section}

%\sectionnotes{1 lecture\EPref{, \S9.6 dans \cite{EP}}\BDref{,
%\S10.7 dans \cite{BD}}}
%
Imaginez avoir une corde de guitare tendue de longueur $ L $.  On considère les vibrations dans une seule direction.  
Soit $ x $  la position le long de la chaîne,  soit $ t $ le temps,  et soit $ y $ le déplacement de la corde depuis la position de repos.
Regardons
\figurevref{we:vibstrfig}.

\begin{myfig}
\capstart
\inputpdft{sps-vibstr}
\caption{Chaîne vibrante de longueur $ L $,  où $ x $ est la position et $ y $ est le déplacement.\label{we:vibstrfig}}
\end{myfig}

L'équation qui régit cette configuration se nomme
\emph{\myindex{équation d'onde unidimensionnelle}}\index{wave equation}:
\begin{equation*}
\mybxbg{~~
y_{tt} =
a^2 y_{xx} ,
~~}
\end{equation*}
pour une certaine constante $a > 0$.
L'intuition est similaire à l'équation de la chaleur. En remplaçant la vitesse par
l'accélération: l'accélération en un point spécifique qui est proportionnel à la seconde
dérivé de la forme de la corde.  En d'autres termes,
quand la chaîne est
concave alors $u_{xx}$ est négatif et la chaîne veut accélérer
vers le bas,  donc $u_{tt}$ devrait être négatif.  Et vice versa.
L'équation d'onde est un exemple d'ÉDP hyperbolique.

Supposons que les extrémités de la corde soient fixées comme sur la guitare:
\begin{equation*}
y(0,t) = 0 \qquad \text{et} \qquad y(L,t) = 0.
\end{equation*}
Notez qu'on a deux conditions le long de l'axe $ x $, car il y a
deux dérivés dans la direction $ x $.

Il existe également deux dérivés le long de la direction $ t $.  On a donc besoin
deux autres conditions ici.  On doit connaître la position initiale
et la vitesse initiale de la corde.  Pour certaines fonctions $ f (x) $ et $ g (x) $, on impose
\begin{equation*}
y(x,0) = f(x)  \qquad \text{et} \qquad y_t (x,0) =
g(x) .
\end{equation*}

L'équation est linéaire,  donc la superposition fonctionne exactement comme pour l'équation de la chaleur.  Encore une fois, on utilise la séparation de variables pour trouver
suffisamment de solutions élémentaires afin d'obtenir la solution générale.  Il y a
un changement cependant.  Il sera plus facile de résoudre deux problèmes distincts
et d'ajouter leurs solutions.


Les deux problèmes que nous allons résoudre sont
\begin{equation} \label{wave:weq}
\begin{array}{ll}
w_{tt} = a^2 w_{xx} , &  \\
w(0,t) = w(L,t) = 0 , &  \\
w(x,0) = 0 & \qquad \text{pour } \; 0 < x < L , \\
w_t(x,0) = g(x) & \qquad \text{pour } \; 0 < x < L ,
\end{array}
\end{equation}
et
\begin{equation} \label{wave:zeq}
\begin{array}{ll}
z_{tt} = a^2 z_{xx} , &  \\
z(0,t) = z(L,t) = 0 , &  \\
z(x,0) = f(x) & \qquad \text{pour } \; 0 < x < L , \\
z_t(x,0) = 0 & \qquad \text{pour } \; 0 < x < L .
\end{array}
\end{equation}

Le principe de superposition implique que 
$y = w + z$ résolve l'équation d'onde et aussi 
$y(x,0) = w(x,0) + z(x,0) = f(x)$ et
$y_t(x,0) = w_t(x,0) + z_t(x,0) = g(x)$.  Par conséquent,  $y$ est une solution à 

\begin{equation} \label{wave:yeq}
\begin{array}{ll}
y_{tt} = a^2 y_{xx} , &  \\
y(0,t) = y(L,t) = 0 , &  \\
y(x,0) = f(x) & \qquad \text{pour } \; 0 < x < L , \\
y_t(x,0) = g(x) & \qquad \text{pour } \; 0 < x < L .
\end{array}
\end{equation}

La raison de toute cette complexité est que la superposition ne fonctionne que pour les
conditions homogènes telles que
$y(0,t) = y(L,t) = 0$, $y(x,0) = 0$,  ou $y_t(x,0) = 0$.  Par conséquent,
on peut utiliser la séparation de variables pour trouver de nombreux élément appartenant à la
solution résolvant toutes les conditions homogènes.  On peut  ensuite les utiliser pour
construire une solution satisfaisant à la condition non homogène restante.
Commençons avec  \eqref{wave:weq}.
On essaie une solution de la forme $w(x,t) = X(x) T(t)$.  On remplace dans l'équation d'onde et l'on obtient 
\begin{equation*}
X(x)T''(t) = a^2 X''(x) T(t) .
\end{equation*}
En réécrivant,  on obtient 
\begin{equation*}
\frac{T''(t)}{a^2 T(t)} = \frac{X''(x)}{X(x)} .
\end{equation*}
Encore une fois, le côté gauche dépend seulement de $ t $ et le côté droit dépend
seulement de $ x $.  Donc les deux côtés sont égales à une constante que l'on nomme
$-\lambda$:
\begin{equation*}
\frac{T''(t)}{a^2 T(t)} = -\lambda = \frac{X''(x)}{X(x)} .
\end{equation*}
On résolve pour obtenir deux équations différentielles ordinaires
\begin{align*}
X''(x) + \lambda X(x) &= 0 , \\
T''(t) + \lambda a^2 T(t) &= 0 .
\end{align*}
Les conditions $0 = w(0,t) = X(0) T(t)$ impliquent $X(0) = 0$ et
$w(L,t) = 0$ implique que $X(L) = 0$.  Par conséquent,  les seules solutions non triviales
 pour la première équation sont 
$\lambda = \lambda_n = \frac{n^2 \pi^2}{L^2}$ et elles sont
\begin{equation*}
X_n(x) = \sin \left( \frac{n \pi}{L} x \right) .
\end{equation*}
La solution générale pour $T$ pour un $\lambda_n$ particulier est  
\begin{equation*}
T_n(t) = A \cos \left( \frac{n \pi a}{L} t \right)
+ B \sin \left( \frac{n \pi a}{L} t \right).
\end{equation*}
On résout aussi les conditions $w(x,0) = 0$ ou $X(x)T(0) = 0$.  Ce qui implique que $T(0) = 0$,  ce qui force $A = 0$.  Un choix judicieux est $B=\frac{L}{n \pi a}$ (vous verrez pourquoi dans un instant)
et alors
\begin{equation*}
T_n(t) = \frac{L}{n \pi a} \sin \left( \frac{n \pi a}{L} t \right).
\end{equation*}
Les solutions de base sont
\begin{equation*}
w_n(x,t) = 
\frac{L}{n \pi a} 
\sin \left( \frac{n \pi}{L} x \right)
\sin \left( \frac{n \pi a}{L} t \right) .
\end{equation*}
On différentie par rapport à $t$:
\begin{equation*}
\frac{\partial w_n}{\partial t}(x,t) = 
\sin \left( \frac{n \pi}{L} x \right)
\cos \left( \frac{n \pi a}{L} t \right) .
\end{equation*}
Ainsi,
\begin{equation*}
\frac{\partial w_n}{\partial t}(x,0) =
\sin \left( \frac{n \pi}{L} x \right) .
\end{equation*}
On étend $g(x)$ en termes de ces sinus:
\begin{equation*}
g(x) =
\sum_{n=1}^\infty b_n \sin \left( \frac{n \pi}{L} x \right) .
\end{equation*}
En utilisant la superposition
on écrit la solution à \eqref{wave:weq} comme une série
\begin{equation*}
w(x,t) =
\sum_{n=1}^\infty
b_n
w_n(x,t)
=
\sum_{n=1}^\infty
b_n
\frac{L}{n \pi a}
\sin \left( \frac{n \pi}{L} x \right)
\sin \left( \frac{n \pi a}{L} t \right) .
\end{equation*}

\begin{exercise}
Vérifiez que $w(x,0) = 0$ et
$w_t(x,0) = g(x)$.
\end{exercise}

On résout \eqref{wave:zeq} similarly.  On essaie encore
$z(x,y) = X(x)T(t)$.  
La procédure fonctionne exactement de la même manière qu'avant.
On obtient
\begin{align*}
X''(x) + \lambda X(x) &= 0 , \\
T''(t) + \lambda a^2 T(t) &= 0 ,
\end{align*}
et les conditions $X(0) = 0$, $X(L) = 0$.  Alors encore une fois, 
$\lambda = \lambda_n = \frac{n^2 \pi^2}{L^2}$ et
\begin{equation*}
X_n(x) = \sin \left( \frac{n \pi}{L} x \right) .
\end{equation*}
Cette fois,  la condition sur $T$ est $T'(0) = 0$.  Ainsi, on obtient que $B = 0$ et l'on prend
\begin{equation*}
T_n(t) = \cos \left( \frac{n \pi a}{L} t \right).
\end{equation*}
Notre solution de base est
\begin{equation*}
z_n(x,t) = 
\sin \left( \frac{n \pi}{L} x \right)
\cos \left( \frac{n \pi a}{L} t \right) .
\end{equation*}
Comme $z_n(x,0) = \sin \left( \frac{n \pi}{L} x \right)$,
on étend $f(x)$ en termes de ces sinus:
\begin{equation*}
f(x) =
\sum_{n=1}^\infty c_n \sin \left( \frac{n \pi}{L} x \right) .
\end{equation*}
Et l'on écrit la solution à \eqref{wave:zeq} comme une série
\begin{equation*}
z(x,t) =
\sum_{n=1}^\infty
c_n
z_n(x,t)
=
\sum_{n=1}^\infty
c_n
\sin \left( \frac{n \pi}{L} x \right)
\cos \left( \frac{n \pi a}{L} t \right) .
\end{equation*}

\begin{exercise}
Remplissez les détails de la dérivé de la solution de \eqref{wave:zeq}.
Vérifiez que la solution satisfait à toutes les conditions secondaires.
\end{exercise}

En réunissant ces deux solutions,  on obtient le théorème  suivant. 
\begin{theorem}
Soit l'équation
\begin{equation} \label{wave:tyeq}
\begin{array}{ll}
y_{tt} = a^2 y_{xx} , &  \\
y(0,t) = y(L,t) = 0 , &  \\
y(x,0) = f(x) & \qquad \text{pour } \; 0 < x < L , \\
y_t(x,0) = g(x) & \qquad \text{pour } \; 0 < x < L ,
\end{array}
\end{equation}
où
\begin{equation*}
f(x) =
\sum_{n=1}^\infty c_n \sin \left( \frac{n \pi}{L} x \right)
\qquad \text{et} \qquad
g(x) =
\sum_{n=1}^\infty b_n \sin \left( \frac{n \pi}{L} x \right) .
\end{equation*}
Alors la solution $ y (x, t) $ peut être écrite comme une somme des solutions
de \eqref{wave:weq} et de  \eqref{wave:zeq}:
\begin{equation*}
\mybxbg{~~
\begin{aligned}
y(x,t)
& =
\sum_{n=1}^\infty
b_n
\frac{L}{n \pi a}
\sin \left( \frac{n \pi}{L} x \right)
\sin \left( \frac{n \pi a}{L} t \right) 
+
c_n
\sin \left( \frac{n \pi}{L} x \right)
\cos \left( \frac{n \pi a}{L} t \right) 
\\
& =
\sum_{n=1}^\infty
\sin \left( \frac{n \pi}{L} x \right)
\left[
b_n
\frac{L}{n \pi a}
\sin \left( \frac{n \pi a}{L} t \right) 
+
c_n
\cos \left( \frac{n \pi a}{L} t \right) 
\right] .
\end{aligned}
~~}
\end{equation*}
\end{theorem}

\begin{example} \label{example:pluckedstring}
Considérons une corde de longueur 2 qui est pincée au milieu.
Elle a une forme initiale donnée dans \ figurevref {wave: pluckedstrfig}.
C'est,
\begin{equation*}
f(x) = \begin{cases}
0.1\, x & \text{si } \; 0 \leq x \leq 1 , \\
0.1\, (2-x) & \text{si } \; 1 < x \leq 2 .
\end{cases}
\end{equation*}

\begin{myfig}
\capstart
\inputpdft{wave-pluckedstr}
\caption{Forme initiale d'une corde pincée, comme à 
l'\exampleref{example:pluckedstring}.\label{wave:pluckedstrfig}}
\end{myfig}

Laissons la corde commencer à la position de repos ($g(x) =
0$), et laissons $a=1$ pour simplifier les choses.  En d'autres termes,  on souhaite
résoudre le problème:
\begin{align*}
& y_{tt} = y_{xx}, \\
& y(0,t) = y(2,t)= 0 , \\
& y(x,0) = f(x) \quad \text{et} \quad y_t(x,0)= 0 .
\end{align*}

On laisse au lecteur le soin de calculer la série sinusoïdale de $ f (x) $.  La série
sera
\begin{equation*}
f(x) = \sum_{n=1}^\infty
\frac{0.8}{n^2 \pi^2}
\sin \left( \frac{n \pi}{2} \right)
\sin \left( \frac{n \pi}{2} x \right) .
\end{equation*}
On note que
$\sin \left( \frac{n \pi}{2} \right)$
est la séquence $1, 0, -1, 0, 1, 0, -1, \ldots$
for $n = 1,2,3,4,\ldots$.  Ainsi,
\begin{equation*}
f(x) = 
\frac{0.8}{\pi^2}
\sin \left( \frac{\pi}{2} x \right)
-
\frac{0.8}{9 \pi^2}
\sin \left( \frac{3 \pi}{2} x \right)
+
\frac{0.8}{25 \pi^2}
\sin \left( \frac{5 \pi}{2} x \right)
- \cdots
\end{equation*}
La solution $y(x,t)$ est donnée par
\begin{equation*}
\begin{split}
y(x,t) & = 
\sum_{n=1}^\infty
\frac{0.8}{n^2 \pi^2}
\sin \left( \frac{n \pi}{2} \right)
\sin \left( \frac{n \pi}{2} x \right)
\cos \left( \frac{n \pi}{2} t \right)
\\
& = 
\sum_{m=1}^\infty
\frac{0.8 {(-1)}^{m+1}}{{(2m-1)}^2 \pi^2}
\sin \left( \frac{(2m-1) \pi}{2} x \right)
\cos \left( \frac{(2m-1) \pi}{2} t \right)
\\
& =
\frac{0.8}{\pi^2} 
\sin \left( \frac{\pi}{2}  x \right)
\cos \left( \frac{\pi}{2}  t \right)
-
\frac{0.8}{9 \pi^2} 
\sin \left( \frac{3 \pi}{2}  x \right)
\cos \left( \frac{3 \pi}{2}  t \right)
\\
& \hspace{20em}
+
\frac{0.8}{25 \pi^2}
\sin \left( \frac{5 \pi}{2}  x \right)
\cos \left( \frac{5 \pi}{2}  t \right) 
- \cdots
\end{split}
\end{equation*}

Regardons la 
\figureref{wave:pluckedexfig} pour un graphique où
 $0 < t < 3$.  On remarque 
que contrairement à l'équation de la chaleur,  la solution ne devient pas
\myquote{plus lisse}.  On verra la raison de ce comportement dans la
section suivante où on dérira la solution de l'équation d'onde d'une autre
façon.

\begin{myfig}
\capstart
\diffyincludegraphics{width=5in}{width=7.5in}{wave-pluckedex}
\caption{Forme de la corde pincée pour $0 < t < 3$.\label{wave:pluckedexfig}}
\end{myfig}

Assurez-vous de bien comprendre ce qu'est l'intrigue dans la figure.  Pour chaque $ t $ fixe,  on peut penser à la fonction
$ y (x, t) $ comme une fonction de $ x $ uniquement.  Cette fonction donne la forme de la corde
au temps $ t $.  Regardons la  \figurevref{wave:pluckedtsfig} pour le graphique de $ y $ en fonction de $ x $ à plusieurs valeurs différentes de $ t $.
Sur ce graphique,  on peut voir que ce n'est pas lisse c'est correct.  

\begin{myfig}
\capstart
%original files wave-plucked-slicet0 wave-plucked-slicet0p4 wave-plucked-slicet0p8 wave-plucked-slicet1p2
\diffyincludegraphics{width=6.24in}{width=9in}{wave-plucked-t0-t0p4}
\\[5pt]
\diffyincludegraphics{width=6.24in}{width=9in}{wave-plucked-t0p8-t1p2}
\caption{La corde pincée pour $t=0$, $t=0.4$, $t=0.8$,  et
$t=1.2$.%
\label{wave:pluckedtsfig}}
\end{myfig}
\end{example}

Une chose à retenir est le son d'une guitare.  On remarque que
les fréquences (angulaires) qui apparaissent dans la solution sont
$n \frac{\pi a}{L}$.  Autrement dit, il existe une certaine base
\emph{\myindex{ffréquence fondamentale}} $\frac{\pi a}{L}$,  et l'on obtient aussi
tous les multiples de cette fréquence,  qui en musique sont appelés
la\emph{\myindex{harmoniques}}.  Ces harmoniques qui apparaissent avec une amplitude sont 
appelées par les musiciens des  \emph{\myindex{timbre}} de la note.
Les mathématiciens appellent généralement cela le \emph{\myindex{spectre}}.
Parce que toutes les fréquences sont des multiples d'une fréquence (la fondamentale),
on obtient un beau son agréable.

La fréquence fondamentale $\frac{\pi a}{L}$ augmente lorsqu'on diminue la longueur $ L $.  Autrement dit,  si
on place un doigt sur la touche puis qu'on pince la corde,  on obtient un son plus aigü.
On remarque que la constante $ a $ est donnée par
\begin{equation*}
a = \sqrt{\frac{T}{\rho}} ,
\end{equation*}
où $T$ est la tension et  $\rho$ est la densité linéaire de la corde.
Lorsqu'on serre la corde (on tourne la cheville d'accord d'une guitare) augmente $ a $ et
produit une fréquence fondamentale plus élevée (une note plus aigüe).
Par contre, en utilisant une corde plus lourde,  on
réduit $ a $ et l'on produit une fréquence fondamentale plus basse (une note plus grave).
Une guitare basse a des cordes plus longues et plus épaisses,  tandis qu'un ukulélé a des cordes courtes
en matériau plus léger.

Quelque chose d'assez intéressant est la presque symétrie entre l'espace et le temps.
Dans sa forme la plus simple,  on voit cette symétrie dans les solutions
\begin{equation*}
\sin \left( \frac{n \pi}{L} x \right)
\sin \left( \frac{n \pi a}{L} t \right)  .
\end{equation*}
Sauf pour le $ a $, le temps et l'espace sont identiques.

En général, la solution pour un $ x $ fixe est une série de Fourier en $ t $,  pour
un $ t $ fixe c'est une série de Fourier en $ x $,  et les coefficients sont liés.
Si la forme de $ f (x) $ ou la vitesse initiale ont beaucoup de points anguleux,  alors
l'onde sonore aura beaucoup de points anguleux.  Ce qui est dû au fait que les coefficients de Fourier
de forme initiale descendes à zéro (comme $n \to \infty$) au même rythme que les coefficients de Fourier
de l'onde dans le temps (pour un certain $ x $ fixe).  Donc,  si l'on utilise un objet pointu pour
pincer la corde,  on obtient un son plus net avec une fréquence beaucoup de haute, tandis que si l'on utilise notre pouce,  on obtient un son plus doux.  De même si pn s'approche du pont,  on obtient un 
un son plus net.

En fait,  si l'on regarde la formule de la solution,  on voit que pour tout
 $ x $ fixe, on obtient une série de Fourier presque arbitraire pour $ t $,  à l'exception du 
terme constant.  On peut essentiellement obtenir le son de notre choix
en pinçant la corde de la bonne manière.
Bien sûr,  on  envisage une corde idéale sans rigidité et sans la résistance de l'air.  Ces variables ont clairement un impact sur le son également.

\subsection{Exercises}

\begin{exercise}
Résolvez
\begin{equation*}
\begin{array}{ll}
y_{tt} = 9 y_{xx} , &  \\
y(0,t) = y(1,t) = 0 , &  \\
y(x,0) = \sin (3\pi x) + \frac{1}{4} \sin (6 \pi x) & \qquad \text{pour } \; 0 < x < 1 , \\
y_t(x,0) = 0 & \qquad \text{pour } \; 0 < x < 1 .
\end{array}
\end{equation*}
\end{exercise}

\begin{exercise}
Résolvez
\begin{equation*}
\begin{array}{ll}
y_{tt} = 4 y_{xx} , &  \\
y(0,t) = y(1,t) = 0 , &  \\
y(x,0) = \sin (3\pi x) + \frac{1}{4} \sin (6 \pi x) & \qquad \text{pour } \; 0 < x < 1 , \\
y_t(x,0) = \sin (9 \pi x) & \qquad \text{pour } \; 0 < x < 1 .
\end{array}
\end{equation*}
\end{exercise}

\begin{exercise}
Dérivez la solution pour une corde pincée de longueur $ L $ et pour
une constante $ a $ (dans l'équation $y_{tt} = a^2 y_{xx}$),  où on soulève la corde  à une certaine distance $ b $, au milieu, et on relâche.
\end{exercise}

\begin{samepage}
\begin{exercise}
Imaginez qu'un instrument de musique à cordes tombe sur le sol.  Supposez que
la longueur de la corde est égale à 1 et que $ a = 1 $.  Quand l'instrument de musique frappe
le sol,  la corde est en position de repos et donc $ y (x, 0) = 0 $.  Cependant,
la corde se déplaçait à une certaine vitesse à l'impact ($ t = 0 $),
c'est-à-dire $ y_t (x, 0) = -1 $.  Trouvez la solution
$ y (x, t) $ pour la forme de la corde au temps $ t $.
\end{exercise}
\end{samepage}

\begin{exercise}[défi]
Supposez avoir une corde vibrante avec une résistance de l'air proportionnelle à sa vitesse.  Autrement dit,  vous avez

\begin{equation*}
\begin{array}{ll}
y_{tt} = a^2 y_{xx} - k y_t , &  \\
y(0,t) = y(1,t) = 0 , &  \\
y(x,0) = f(x) & \qquad \text{pour } \; 0 < x < 1 , \\
y_t(x,0) = 0 & \qquad \text{pour } \; 0 < x < 1 .
\end{array}
\end{equation*}
Supposez que $0 < k < 2 \pi a$.
Dérivez une solution au problème en série. Tous les coefficients de la série
doivent être exprimés comme des intégrales de $f(x)$.
\end{exercise}

\begin{exercise}
Supposez que vous touchez à la corde d'une guitare exactement au milieu pour
assurer la condition $u(\nicefrac{L}{2},t) = 0$ en tout temps.
Quels multiples de la fréquence fondamentale $\frac{\pi a}{L}$
apparaissent dans la solution?
\end{exercise}

\setcounter{exercise}{100}

\begin{exercise}
Résolvez
\begin{equation*}
\begin{array}{ll}
y_{tt} = y_{xx} , &  \\
y(0,t) = y(\pi,t) = 0 , &  \\
y(x,0) = \sin(x) & \qquad \text{pour } \; 0 < x < \pi , \\
y_t(x,0) = \sin(x) & \qquad \text{pour } \; 0 < x < \pi .
\end{array}
\end{equation*}
\end{exercise}
\exsol{%
$
y(x,t)
=
\sin(x)
\bigl(\sin(t) + \cos(t)\bigr)
$
}

\begin{exercise}
Résolvez
\begin{equation*}
\begin{array}{ll}
y_{tt} = 25 y_{xx} , &  \\
y(0,t) = y(2,t) = 0 , &  \\
y(x,0) = 0 & \qquad \text{pour } \; 0 < x < 2 , \\
y_t(x,0) = \sin(\pi t) + 0.1 \sin(2\pi t) & \qquad \text{pour } \; 0 < x < 2 .
\end{array}
\end{equation*}
\end{exercise}
\exsol{%
$y(x,t)
=
\frac{1}{5 \pi}
\sin (\pi x)
\sin (5 \pi t)
+
\frac{1}{100 \pi}
\sin(2\pi x)
\sin(10\pi t)$
}
%$
%y(x,t)
%=
%\sum\limits_{n=1}^\infty
%\sin \left( \frac{n \pi}{L} x \right)
%\left[
%b_n
%\frac{L}{n \pi a}
%\sin \left( \frac{n \pi a}{L} t \right) 
%+
%c_n
%\cos \left( \frac{n \pi a}{L} t \right) 
%\right] .

\begin{exercise}
Résolvez
\begin{equation*}
\begin{array}{ll}
y_{tt} = 2 y_{xx} , &  \\
y(0,t) = y(\pi,t) = 0 , &  \\
y(x,0) = x & \qquad \text{pour } \; 0 < x < \pi , \\
y_t(x,0) = 0 & \qquad \text{pour } \; 0 < x < \pi .
\end{array}
\end{equation*}
\end{exercise}
\exsol{%
$
y(x,t)
=
\sum\limits_{n=1}^\infty
\frac{2{(-1)}^{n+1}}{n}
\sin(nx)
\cos( n \sqrt{2}\,t ) 
$
%$
%y(x,t)
%=
%\sum\limits_{n=1}^\infty
%\sin \left( \frac{n \pi}{L} x \right)
%\left[
%b_n
%\frac{L}{n \pi a}
%\sin \left( \frac{n \pi a}{L} t \right) 
%+
%c_n
%\cos \left( \frac{n \pi a}{L} t \right) 
%$
}

\begin{exercise}
Regardez ce qui arrive lorsque $a=0$.  Trouvez la solution à
$y_{tt} = 0$, $y(0,t) = y(\pi,t) = 0$,
$y(x,0) = \sin(2x)$,
$y_t(x,0) = \sin(x)$.
\end{exercise}
\exsol{%
$y(x,t) = \sin(2x)+t\sin(x)$
}

